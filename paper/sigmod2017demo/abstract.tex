\begin{abstract}

\eat{
Graphs are used to represent a plethora of phenomena, from the Web and
social networks, to biological pathways, to semantic knowledge
bases. Arguably the most interesting and important questions one can
ask about graphs have to do with their evolution. Which Web pages are
showing an increasing popularity trend? How does influence propagate
in social networks? How does knowledge evolve?  
}

Evolving graphs are used to represent evolution of common phenomena,
from the Web and social networks to biological pathways.  Analysis
tasks on evolving graphs include computing change of common graph
measures (influence, centrality, connectivity), zooming in and out on
measures of interest by modifying time granularity, and detecting
temporal patterns.

In this demonstration we present Portal, a system for efficient
querying and exploratory analysis of evolving graphs, built on Apache
Spark. The Portal system implements a declarative query language on
top of a compositional temporal graph algebra \tga. We provide a
text-based query shell where the user can compose and execute queries,
as well as inspect query plans.  We demonstrate that our system is
usable and efficient, by showing that sophisticated kinds of
exploratory analysis of large evolving graphs can be expressed
intuitively and concisely, and executed in interactive time.

\end{abstract}
