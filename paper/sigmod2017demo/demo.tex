\section{Demonstration Details}
\label{sec:demo}

We will present an end-to-end implementation of \sys and will
demonstrate that our system is efficient and usable, by showing that
sophisticated kinds of exploratory analysis of large evolving graphs
can be expressed intuitively and concisely, and executed in
interactive time.

\eat{
\subsection{Demo Setup}
\label{sec:setup}}

Users will interact with the \sys system via an interactive shell,
where they will compose queries and define \tg views, exploring
language features.  Users will inspect optimized query execution plans
and execute queries.

We will have a local cluster on a single laptop with three evolving
graph datasets that allow interactive-speed data exploration. DBLP
contains co-authorship information from 1936 through 2015, with 2.4
million author nodes and 7.2 million undirected co-authorship edges.
arXiv contains co-authorship information from 1993 through 2016,
filtered to Computer Science publications with 0.26 million user nodes
and 0.4 million messaging edges.  wiki-talk contains over 10 million
messaging events among 3 million wiki-en users\eat{2002 through 2015},
aggregated at 1-month resolution.

\eat{
\subsection{Story Line}
\label{sec:story}}

\eat{We will demonstrate the functionality of \sys by executing queries
provided by us and by members of the audience. }

The demonstration will consist of three parts.  In Part 1, we will
offer an introduction to the \ql operations and syntax.  In Part 2, we
will showcase the queries from Section~\ref{sec:cases}.  Each of the
queries is expected to execute in interactive time.  \eat{ take on the
  order of 30 seconds or less.  Queries on very large evolving graphs
  will not result in interactive running times on the stand-alone
  cluster we plan to use for the demonstration.  However, the DBLP and
  arXiv datasets that we will primarily use allow for reasonable
  response time. }We will examine the query execution plans, including
the order of execution of \tga operators and the selected access
methods.  In Part 3, we will encourage the audience to use the
interactive shell and define new analyses on the demo data sets.
