\section{Related Work}
\label{sec:related}

Three systems in the literature focus on systematic support of
evolving graphs, all of them non-compositional.  Miao et
al.~\cite{Miao2015} developed ImmortalGraph (formerly Chronos), a
proprietary in-memory execution engine for temporal graph analytics.
ImmortalGraph does not provide a query language, focusing primarily on
efficient physical data layout.  The batching method for snapshot
analytics we use is similar to the one proposed in ImmortalGraph.
However, ImmortalGraph was developed with the focus on centralized
rather than distributed computation and~\cite{Miao2015} does not
explore the effect of distribution on batching performance.

The G* system~\cite{Labouseur2015} manages graphs that correspond to
periodic snapshots, with the focus on efficient data layout.  It takes
advantage of the similarity between successive snapshots by storing
shared vertices only once and maintaining per-graph indexes.  Time is
not an intrinsic part of the system, as there is in \tga, and thus
temporal queries with time predicates like node creation are not
supported.  G* provides two query languages: procedural query language
PGQL, and a declarative graph query language (DGQL). PGQL provides
graph operators such as retrieving vertices and their edges from disk
and non-graph operators like aggregate, union, projection, and join.
All operators use a streaming model, i.e. like in traditional DBMS,
they pipeline.  DGQL is similar to SQL and is converted into PGQL by
the system.

Finally, the Historical Graph Store (HGS) system is an evolving graph
query system based on Spark.  It uses the property graph model and
supports retrieval tasks along time and entity dimensions through Java
and Python API.  It provides a range of operators such as selection
(equivalent to our subgraph operators but with no temporal
predicates), timeslice, nodecompute (similar to map but also with no
temporal information), as well as various evolution-centered
operators.  HGS does not provide formal semantics for any of the
operations it supports and the main focus is on efficient on-disk
representation for retrieval.

None of the three systems are publicly available, so direct
performance comparison with them is not feasible.

