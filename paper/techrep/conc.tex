\section{Conclusions and Future Work}
\label{sec:conc}

In this paper we presented \ql, a declarative query language for
evolving graphs.  We also proposed an implementation of \ql in scope
of Apache Spark, a distributed open-source processing framework.  We
implemented several physical representations of evolving graphs, and
several partitioning strategies, and studied their relative
performance for \ql operators.  

Our experiments demonstrate interesting trade-offs between spatial and
temporal locality.  This work opens many avenues for future work.  It
is in our immediate plans to start work on a query optimizer for \ql.
We will also implement and evaluate additional \tg representations
that explore the trade-off between density and compactness, and
between temporal and structural locality.  Finally, we are working on
extending the class of trend analytics, and on optimizing the
performance of snapshot and trend analytics.

\eat{ While the Portal language is extensive, it is by no means
  complete. We recognize that there are a number of operations that
  are currently not supported but would be useful to potential users:}

\eat{\begin{enumerate}
\item Temporal pattern matching.  While aggregation provides an
  ability to detect some patters, a more general temporal-structural
  pattern mining is needed.  Work on specifying structural patterns in
  graphs is ongoing, however, at the time of this writing we are not
  aware of a general approach for specifying structural patterns that
  also have a time dimension.  For example, how should the user
  specify that he/she is looking for small strongly connected
  components that exhibit consistent growth over a period of time?
  Some work on this front has been done by Chan et
  al.~\cite{Chan2008,Kan2009}.
\item Structural select (subgraph). 
\item Other kinds of temporal select besides by interval, such as with
  predicates, similar to snapshot selection support
  in~\cite{Khurana2013}. 
\item Across-time analytics (unlike snapshot-based analytics,
  e.g. pagerank) like the centrality metric for dynamic networks,
  where influence of a vertex propagates through time.
\item Anything that would return not another tgraph.  This could be
  measures of a whole graph (measure of centrality, degree
  distribution, diameter, etc.) or searching that returns a set of tgraphs,
  e.g. frequent pattern mining).
\end{enumerate}}
