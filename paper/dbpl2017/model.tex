\section{Temporal Graph Model}
\label{sec:model}

Our data model represents a single evolving graph, consisting of
nodes, node properties, edges, edge properties, and addition,
modification, and deletion of these over time.  It uses a closed world
semantics -- we assume that evolution history is complete, the state
of the graph is fully known and recorded at each time instant, and
there is no uncertain or incomplete information.

\subsection{Preliminaries}

We assume a linearly ordered, discrete time domain $\Omega^T$ where
time {\em instances} have limited precision.  In temporal relational
databases, a {\em valid-time} temporal relation schema is represented
as $R = (A_1, \ldots, A_m~|~T)$, where $A_1, \ldots, A_m$ are
nontemporal attributes with domain $\Omega_i$ and $T$ is a temporal
attribute over $\Omega^T \times \Omega^T$.  The timestamp attribute is
special and thus separated by a $|$ symbol in the list of attributes.
\eat{It is also always listed last.  }This is called {\em tuple
  timestamping}~\cite{Montanari2009}, since each tuple in a relation
is associated with a single time attribute during which it is known to
hold.  In principle, the time attribute can be a single time instant
or a set of instants.  We use periods to compactly represent the
constituent time points.  This is a common representation technique,
which does not add expressive power to the data model, compared to
associating each tuple with a single time
instant~\cite{DBLP:conf/ictl/Chomicki94}.  Following the SQL:2011
standard~\cite{DBLP:journals/sigmod/KulkarniM12}, a period (or
interval) $t = [s, e)$ represents a discrete contiguous set of time
  instances from domain $\Omega^T$, starting from and including the
  start time $s$, continuing to but excluding the end time $e$.

With point semantics, a temporal relation is required to be coalesced
-- there are no value-equivalent tuples with consecutive timestamps.
The statement that a relation is coalesced means that each fact is
represented once for each time period of maximal length when it
holds~\cite{DBLP:conf/vldb/BohlenSS96}.  Requiring that relations be
coalesced is both space-efficient and avoids semantic
ambiguity~\cite{DBLP:reference/db/JensenS09k}.

\subsection{TGraph}

We now describe the logical representation of an evolving graph,
called a \tg.  A \tg represents a single graph, and models the
evolution of its topology and of vertex and edge properties.  \eat{This is
in line with the nontemporal graph definitions.  }

In order to represent evolving graphs, we provide two definitions: 1)
an extension of the property graph
definition~\cite{DBLP:journals/corr/AnglesABHRV16}, and 2) one that is
based on the temporal relataional model.  The two models are equivalent,
but it is sometimes easier to define the semantics of an operation
over one definition than the other.  We extend the property graph
model to represent graph nodes and edges as follows: each node and
edge is associated with a period of validity, as is each property
value.\eat{during time $t$ is associated with a (possibly empty) {\em
    set} of properties, and each property is represented by a
  key-value pair.}  A \tg can also be represented with two nested
temporal relations, associating a fact (existence of a vertex or an edge,
and an assignment of a value to a vertex or an edge attribute) with a
time period.

\begin{table}
\centering
\setlength\tabcolsep{1.8pt}
\caption{A co-authorship network represented using the \tg model,
  consisting of two nested temporal relations.}
\vspace{-0.2cm}
\label{tab:tg}
\centering
%\begin{tabular}{| c | c | c | c | c |}
\begin{tabular*}{\columnwidth}{@{\extracolsep{\fill}}|c|c|c|p{1.84in}|c|}
\hline
\multicolumn{5}{|l|}{$TV$} \\
\multicolumn{3}{|c}{\bfseries{\underline v}} & \multicolumn{1}{c}{\bfseries a} & \multicolumn{1}{c|}{\bfseries T}\\ \hline
\multicolumn{3}{|c|}{v1} & type$\shorteq$p,name$\shorteq$Alice,school$\shorteq$Drexel & ['15/1,'15/7) \\ \hline
\multicolumn{3}{|c|}{v2} & type$\shorteq$p,name$\shorteq$Bob & ['15/2,'15/5) \\ \hline
\multicolumn{3}{|c|}{v2} & type$\shorteq$p,name$\shorteq$Bob,school$\shorteq$CMU & ['15/5,'15/10) \\ \hline
\multicolumn{3}{|c|}{v3} & type$\shorteq$p,name$\shorteq$Cathy,school$\shorteq$Drexel & ['15/1,'15/10) \\ \hline
%\multicolumn{5}{|l|}{} \\
\multicolumn{5}{|l|}{$TE$} \\
\multicolumn{1}{|c}{\bfseries{\underline e}} & \multicolumn{1}{c}{\bfseries v1} & \multicolumn{1}{c}{\bfseries v2} & \multicolumn{1}{c}{\bfseries a} & \multicolumn{1}{c|}{\bfseries T} \\ \hline
e1 & v1 & v2 & type$\shorteq$co-author,cnt$\shorteq$3 & ['15/2,'15/6) \\ \hline
e2 & v2 & v3 & type$\shorteq$co-author,cnt$\shorteq$4 & ['15/7,'15/10) \\ \hline
\end{tabular*}
\vspace{-0.2cm}
\end{table}

Table~\ref{tab:tg} gives an example of a \tg that shows evolution of a
co-authorship network in the nested temporal relational model.  Node
\insql{1} persists without change from '15/1 to '15/7, but node
\insql{2} has a property change event at '15/5, thus creating a new
tuple.  For the sake of readability we display the nested attribute
$a$ as a list of key value pairs.

We now give a formal definition of a \tg.

\begin{definition}[TGraph]
\label{def:tg1}
A \tg $\tve$ is a 7-tuple \break $(V,E,\Pi,\rho,\xi^T,\lambda^T_v,\lambda^T_e)$, where:

\begin{itemize}[noitemsep,topsep=3pt,itemindent=\dimexpr\labelwidth+\labelsep\relax,leftmargin=7pt]
\item $V$ is a finite set of {\em nodes},
\item $E$ is a finite set of {\em edges},
\item $\Pi$ is a set of available properties, where each element in $\Pi$ is a pair $(L_i,D_i)$, $L_i$ is a property name, and $D_i$ is the domain of property $L_i$,
\item $\rho: E \to (V \times V)$ is a total function that maps an edge to its source and destination node,
\item $\xi^T: (V \cup E) \times \Omega^T \times \Omega^T \to B$ is a total function that maps a node or an edge and time period to a boolean, indicating whether the node (resp. edge) existed during that whole time period,
\item $\lambda^T_v: V \times \bigcup L_i \times \Omega^T \times \Omega^T \to Val$ is a partial function that maps a node from $V$ and a property label from $(L_i,D_i)$ pairs in $\Pi$ during time period to a value in domain $D_i$ at that time,
\item $\lambda^T_e: E \times \bigcup L_i \times \Omega^T \times \Omega^T \to Val$ is a partial function that maps an edge in $E$ and a property label from $(L_i,D_i)$ pairs in $\Pi$ during a time period to a value of that property in its domain $D_i$ at that time.
\end{itemize}
\end{definition}

This definition is similar to the static graph
definition~\cite{DBLP:journals/corr/AnglesABHRV16}, with the addition
of the $\xi^T$ function that associates nodes and edges with their
periods of validity, and the modification of the $\lambda$ functions
that map from entities to their property values, to include time.
Properties can only take on a value during time periods when the node
(resp. edge) exists.  The property set of each node or edge may {\bf
  not} be empty -- we require the \insql{type} property in place of a
label.  For consistency, we restrict property domains to atomic types
and those that can be represented by a V-relation.

As in temporal relational algebra (TRA), time is part of the model
rather than an additional property in the property set.  Time is
treated special and cannot be modified by the user directly, although
it can be accessed in predicates.  \eat{This is one of the lessons
  from the temporal relational work and we incorporate it here.} Note
that the $\rho$ function is not temporal -- an edge always connects
the same two nodes, whenever it exists.

In order to take advantage of \tra, we provide an alternate definition
using two nested temporal relations, assuming, for the
  moment, that all node/edge properties are in a single domain $D$.

\begin{definition}[TGraph-Relational]
\label{def:tg2}
A \tg is a pair $\mathcal{G}=(\tv, \te)$. \tv is a valid-time nested
temporal relation with schema $\tv(\underline{v}, a: (K,B) ~|~T)$,
where $K$ is a property key string, and $B$ is a property value.
\eat{each $B_i, 0 \geq i \geq m$ is a value of a property of domain
  $D_i$, and $m$ is the number of different domains in $\Pi$.}  \tv
associates a vertex and its property set with the time period during
which it is present and unchanged. \te is a valid-time nested temporal
relation with schema $\te(\underline{e}, v_1, v_2, a:
(K,B) ~|~T)$, connecting pairs of vertices from \tv.

Relations of $\mathcal{G}$ must meet the following requirements:

\begin{description}[noitemsep,topsep=3pt,itemindent=\dimexpr\labelwidth+\labelsep\relax,leftmargin=0pt]
\item [R1: Unique vertices/ edges] In every snapshot $\tau_p (\tv)$
  and $\tau_p (\te)$, where $p$ is a time point, a vertex/edge
  exists at most once.  That is, we require set-based semantics with
  duplicate free temporal relations.
\item [R2: Referential integrity] In every snapshot $\tau_p(\te)$
  foreign key constraints hold to $\tau_p (\tv)$ on both $v_1$ and
  $v_2$.
\item [R3: Coalesced] Value-equivalent tuples in all relations of \tve
  with consecutive or overlapping time periods are merged.  For a
  nested attribute, value equivalence is a standard set equivalence,
  i.e., that the sets of property pairs are the same.
\item [R4: Required property] For any $v \in \tv,
  v.a\neq\emptyset$, and for any $e \in \te, e.a\neq\emptyset$.
  That is, we require that each node and edge have at least one
  property (\insql{type}) in its property set.
\item [R5: Constant edge association] For any pair of tuples
  $e_1, e_2 \in \te, e_1.e = e_2.e \implies e_1.v_1 = e_2.v_1 \land
  e_1.v_2 = e_2.v_2$.  That is, consistent with the non-relational
  definition, we require that an edge always connects the same two
  nodes, whenever it exists.
\end{description}
\end{definition}

Property sets are represented by a nontemporal nested relation, and,
more specifically, a V-relation~\cite{Abiteboul1995}.
Definition~\ref{def:tg2} assumes, for simplicity, that all properties
draw values from a single domain.  It is straight-forward to
generalize this definition, allowing the set of property domains to
not be limited to a single domain or even to atomic values.  We can
include any domain that can be represented by an atomic type or a
nested relation.  \eat{Observe that the set of property domains
  $\bigcup B_i$ is not limited to atomic values, but can be a nested
  relation.  Thus sets of atomic values, maps, and similar
  complex-value property types are supported.  }

Requirements {\bf R1 and R2} guarantee soundness of the \tg data
structure, ensuring that every snapshot of a \tg is a valid graph.  If
we remove requirement {\bf R1}, a snapshot at some point $p$ may
contain two instances of the same node or edge, which breaks the
set-based graph semantics.  Requirement {\bf R2} prevents a situation
where an edge connects a node that does not exist at that time
instant.  Requirement {\bf R3} avoids semantic ambiguity and ensures
correctness of algebraic operations in point-stamped temporal models
such as ours~\cite{DBLP:reference/db/JensenS09k}.  Requirement {\bf
  R4} provides compatability with the graph-based definition, where
property \insql{type} is required, and avoids any loss of information
when relations \tv and \te are unnested during operations.

Definitions~\ref{def:tg1} and~\ref{def:tg2} are equivalent.  We prove
this by construction in the Appendix.  When we define operations of
our graph algebra, we will use either \tg or \tg-relational model, as
convenient.
