\section{Introduction}
\label{sec:intro}

Evolving graphs analysis is a rich new area of research that has been
getting increasing
attention~\cite{DBLP:journals/csur/AggarwalS14,Miao2015,Ren2011,Semertzidis2015}.
This is a natural consequence of the prevalence of graph datasets to
represent such wide-ranging phenomena as social networks and the Web.
The phenomena change with time, and the evolution provides another
dimension of interest from which to draw insight.

Given the progress both in the area of graph databases and temporal
databases, it is expected that evolving graphs would be supported as
well, being at the intersection of the two areas.  And yet, systematic
support for scalable querying and analytics over evolving graphs still
lacks, despite much recent interest and activity, and despite increased
variety and availability of evolving graph data.  This support is
urgently needed, due to the scalability and efficiency challenges
inherent in evolving graph analysis, and to considerations of
usability and ease of dissemination.  The goal of our work is to fill
this gap by combining the advances in graph databases and temporal
relational databases.

Previous efforts pursued ad hoc approaches to modeling evolving graphs
by representing time as data or using a snapshot sequence model, with
all inherent limitations.  As a result, while the types of analyses on
evolving graphs are numerous in the literature, there is no unifying
model or a query language.  Previous efforts to provide {\em
  systematic} support have addressed two main areas: efficient
snapshot retrieval~\cite{Khurana2013,Khurana2016} and efficient
snapshot analytics~\cite{Miao2015,MoffittTempWeb16}.  This still
leaves many other kinds of analyses, from discovery of spatio-temporal
patterns, to changing the temporal resolution of the data for a
different look, to combining multiple evolving graph sources into one.

% General motivation
We propose a conceptual representation of an evolving graph, called a
\tg, that captures the evolution of both graph topology and node and
edge attributes (Section~\ref{sec:model}).  This representation builds
on both temporal relational models and graph property models.  To
provide systematic support for analysis of evolving graphs, we propose
a \tg algebra, \tga (Section~\ref{sec:algebra}).  Our goal in
developing \tga is to give users an ability to concisely express a
wide range of common analysis tasks over evolving
graphs~\cite{DBLP:journals/csur/AggarwalS14}.  We present formal
properties of \tga, with a focus on its temporal completeness
(Section~\ref{sec:formal}).  It is the goal of this work to define an
algebra that has clear semantics and is sufficiently expressive to
support evolving graph analysis for a wide audience of data scientists
and researchers.  We illustrate how \tga can be used on several
motivating examples (Section~\ref{sec:usecases}).

%\end{enumerate}


