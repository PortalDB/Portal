\section{Introduction}
\label{sec:intro}

Evolving graphs analysis is a rich new area of research that has been
getting increasing
attention~\cite{DBLP:journals/csur/AggarwalS14,Miao2015,Ren2011,Semertzidis2015}.
This is a natural consequence of the prevalence of graph datasets to
represent such wide-ranging phenomena as social networks and the Web.
The phenomena change with time, and the evolution provides another
dimension of interest from which to draw insight.

Given the progress both in the area of graph databases and temporal
databases, it is expected that evolving graphs would be supported as
well, being at the intersection of the two areas.  And yet, systematic
support for scalable querying and analytics over evolving graphs still
lacks, despite much recent interest and activity, and despite increased
variety and availability of evolving graph data.  This support is
urgently needed, due to the scalability and efficiency challenges
inherent in evolving graph analysis, and to considerations of
usability and ease of dissemination.  The goal of our work is to fill
this gap by combining the advances in graph databases and temporal
relational databases.

Previous efforts pursued ad hoc approaches to modeling evolving graphs
by representing time as data or using a snapshot sequence model, with
all inherent limitations.  As a result, while the types of analyses on
evolving graphs are numerous in the literature, there is no unifying
model or a query language.  Previous efforts to provide {\em
  systematic} support have addressed two main areas: efficient
snapshot retrieval~\cite{Khurana2013,Khurana2016} and efficient
snapshot analytics~\cite{Miao2015,MoffittTempWeb16}.  This still
leaves many other kinds of analyses, from discovery of spatio-temporal
patterns, to changing the temporal resolution of the data for a
different look, to combining multiple evolving graph sources into one.

% General motivation
\eat{
The importance of networks in scientific and commercial domains cannot
be overstated.  Considerable research and engineering effort is being
devoted to developing effective and efficient graph representations
and analytics.  Efficient graph abstractions and analytics for {\em
  static graphs} are available to researchers and practitioners in
scope of open source platforms such as Apache Giraph, Apache Spark /
GraphX~\cite{DBLP:conf/osdi/GonzalezXDCFS14} and GraphLab /
PowerGraph~\cite{DBLP:conf/osdi/GonzalezLGBG12}.
}

\eat{
Arguably the most interesting and important questions one can ask
about networks have to do with their evolution, rather than with their
static state.  Analysis of {\em evolving graphs} has been receiving
increasing
attention~\cite{DBLP:journals/csur/AggarwalS14,Chan2008,Kan2009,Miao2015,Ren2011,Semertzidis2015}.
Yet, despite much recent activity, and despite increased variety and
availability of evolving graph data, systematic support for scalable
querying and analysis of evolving graphs still lacks.  This support is
urgently needed, due first and foremost to the scalability and
efficiency challenges inherent in evolving graph analysis, but also to
considerations of usability and ease of dissemination.  The goal of
our work is to fill this gap.  In this paper, we present an algebraic
query language called \tg algebra, or \tga, and its scalable
implementation in \ql, an open-source distributed framework on top of
Apache Spark.
}

\eat{
Our goal in developing \tga is to give users an ability to concisely
express a wide range of common analysis tasks over evolving graphs,
while at the same time preserving inter-operability with temporal
relational algebra (\tra).  Implementing (non-temporal) graph querying
and analytics in an RDBMS has been receiving renewed
attention~\cite{DBLP:conf/sigmod/AbergerTOR16,DBLP:conf/sigmod/SunFSKHX15,DBLP:journals/pvldb/Xirogiannopoulos15},
and our work is in-line with this trend. Our data model is based on
the temporal relational model, and our algebra corresponds to temporal
relational algebra, but is designed specifically for evolving graphs.
}

\eat{We give a translation of our operators in relational algebra, and, as
a consequence, can implement our algebraic framework on top of a
relational engine.  Th advantages of incorporating graph querying
into a general-purpose relational system are two-fold: usability and
access to the relational data modeling and query processing
infrastructure.}

\eat{For this reason our data model is based on the temporal relational
model, and our query language corresponds to temporal relational
calculus, a two-sorted logic with variables and quantifiers explicitly
ranging over the time domain~\cite{DBLP:reference/db/Toman09}, for
graphs.}

\eat{
We represent graph evolution, including changes in topology and in
attribute values of vertices and edges, using the \tg abstraction ---
a collection of temporal SQL relations with appropriate integrity
constraints.  An example of a \tg is given in Figure~\ref{fig:tg_ve},
where we show evolution of a co-authorship network.
}

%\subsection{Contributions and roadmap}

%
%\begin{enumerate}[noitemsep,leftmargin=*]
%\begin{description}[noitemsep]
%
We propose a conceptual representation of an evolving graph, called a
\tg, that captures the evolution of both graph topology and node and
edge attributes (Section~\ref{sec:model}).  This representation builds
on both temporal relational models and graph property models.  To
provide systematic support for analysis of evolving graphs, we propose
a \tg algebra, \tga (Section~\ref{sec:algebra}).  Our goal in
developing \tga is to give users an ability to concisely express a
wide range of common analysis tasks over evolving
graphs~\cite{DBLP:journals/csur/AggarwalS14}.  We present formal
properties of \tga, with a focus on its temporal completeness
(Section~\ref{sec:formal}).  It is the goal of this work to define an
algebra that has clear semantics and is sufficiently expressive to
support evolving graph analysis for a wide audience of data scientists
and researchers.  We illustrate how \tga can be used on several
motivating examples (Section~\ref{sec:usecases}).

%\end{enumerate}


