\section{Related Work}
\label{sec:related}

{\bf Data model and representation.}  Representation and querying of
evolving graphs has received active interest recently. Much of the
work on evolving graphs is on snapshot retrieval and snapshot
analytics in a discrete time domain.  In this case an evolving graph
is considered to be a sequence of graph snapshots corresponding to
time instances in the time domain $T$
(\cite{Khurana2013,DBLP:journals/tos/MiaoHLWYZPCC15,Ren2011}).  Such a
model has several limitations: a) it requires a discrete time domain.
The state of the graph can be undefined at some time $t_i$ unless a
snapshot is associated with each possible value of $T$ in the discrete
range $[t_{start}, now)$; and b) it is not compact.  Every change to
  an entity (vertex or edge) requires a new snapshot.  For large
  graphs, consecutive snapshots have degree of similarity approaching
  1 (on the scale of 0 of no similarity and 1 full similarity).

Khurana and Deshpande~\cite{Khurana2013} investigate efficient
physical representations using deltas to support snapshot retrieval.
Thus the on-disk representation is compatible with the period-based
model, but the logical model and the algorithms for retrieval are
snapshot-based.  Their in-memory GraphPool maintains a single
representation of all snapshots and stores only dependencies from a
materialized snapshot when deltas are small.  However, an evaluation
of queries involving multiple snapshots, such as aggregation, requires
fully materialized views in memory, which makes this approach
infeasible for our purposes.  Snapshot retrieval, however, is a useful
operation for many types of analysis and thus should be supported.  In
our model a snapshot at any time $t$ is state at time $t$ and can be
retrieved simply by selection on $p$.

Ren et al.~\cite{Ren2011} develop an in-memory representation of
evolving graphs based on representative graphs for sets of snapshots.
Note that this is a different definition of the term {\em
  representative graph} than the one we use, since it indicates a
commonality (a union or an intersection) between a set of consecutive
snapshots, rather than being a consistent representation of the graph
for some period $p$.  Nevertheless, Ren's representative graphs
can be computed in \ql using temporal aggregation and provide further
motivation for the existence of that operation.

Semertzidis et al.~\cite{Semertzidis2015} develop a version graph,
where each node and edge is annotated with the set of time intervals
in which they exist.  Their logical model is also a sequence of
snapshots with the integer time domain.  However, it is easy to show
that the version graph can be extended to support a period-based
model.  Our \og representation is based on this idea.

Boldi et al.~\cite{Boldi2008} present a space-efficient non-delta
approach for storing a large evolving Web graph that they harvested.
Like the work above, it only represents purely topological information
and does not address vertex and edge attributes.  However, a separate
column store could be added for those attributes with relative ease.

Another commonly used graph model in the research literature is a
continuous time model based on change
streams~\cite{Cheng2012,Ediger2012}, usually for the purposes of
supporting analysis of the latest state of the graph quickly and
efficiently.  In this model a stream $S$ emits a sequence of graph
change events $e$, each associated with a time $t$ at which it was
emitted.  An event can be of entity creation, deletion, or change in
the value of one of the entity attributes.  Multiple events may be
associated with one time instance and the event emit rate is not
constant.  Because the stream emits {\em individual entities} rather
than {\em whole graphs}, graph snapshots can only be reconstructed by
maintaining the history of changes over time.  Our evolving graph
model can be constructed from the change stream model following
the conventional temporal database approach of maintaining valid-time data.

\eat{
This approach assumes that no events arrive out of order and there are
no redundant or duplicate events.  However, it would be a simple
extension to relax these two assumptions.  We also use a closed world
assumption -- if no event is recorded, for our purposes it did not
occur.
}

{\bf Querying and analytics.} There has been much recent work on
analytics for evolving graphs,
see~\cite{DBLP:journals/csur/AggarwalS14} for a survey. This line of
work is synergistic with ours, since our aim is to provide systematic
support for scalable querying and analysis of evolving graphs.

Several researchers have proposed individual queries, or classes of
queries, for evolving graphs, but without a unifying syntax or general
framework.  Kan et al.~\cite{Kan2009} propose a query model for
discovering subgraphs that match a specific spatio-temporal pattern.
Chan et al.~\cite{Chan2008} query evolving graphs for patterns
represented by waveforms.  Semertzidis et al.~\cite{Semertzidis2015}
focus on historical reachability queries.

Our work shares motivation with Miao et
al.~\cite{DBLP:journals/tos/MiaoHLWYZPCC15}, who developed an
in-memory execution engine for temporal graph analytics called
ImmortalGraph.  Unlike Miao et al., who focus on in-memory layout and
locality-aware scheduling mechanisms, we work in a distributed
processing environment.  A further difference is that our work is in
scope of Apache Spark, a widely-used open source platform, while
ImmortalGraph is a proprietary stand-alone prototype.
