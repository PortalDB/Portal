\subsection{Temporal User-Defined Analytics}
\label{sec:analytics}

For many types of evolving graph analysis, it is necessary to compute
some property, such as PageRank of each vertex $v$, or the length of
the shortest path from a given designated vertex $u$ to each $v$, for
each representative graph.  This information can then be used to study
how the graph evolves over time.  \ql supports this type of analysis
through {\em temporal user-defined analytics}, which conceptually
execute an analytic over each \rg, compute a value for each vertex of
the \rg, and then store this value among the attributes of the vertex
(adding it to the property bag).  We provide an API that allows
developers to implement custom analytics that can either be computed
locally at a vertex, like degree (these are essentially map
functions~\ref{sec:algebra:map}), or that can be expressed in the
popular Pregel API~\cite{DBLP:conf/sigmod/MalewiczABDHLC10}.

Logically, an analytic is computed on each representative graph.
However, we will show in Section~\ref{sec:exp} that more efficient
methods that use batching are possible.

%$udf(\trg) = \{ (g', p) | g' = udf(g) \cup g\}$

Analytics cannot be computed directly on \tve, because they must, by
definition, act over representative graphs.

{\bf Analytics do not uncoalesce.} An analytic adds a new key-value
pair to the property bag of each vertex $v$ of every representative
graph $g$, and does not implicitly project out vertex attributes.  For
this reason, an analytic cannot cause two temporally adjacent graphs
$g_1$ and $g_2$ that were not equivalent before the new attribute was
added to become equivalent after it was added.

\eat{With bag semantics, if the input was coalesced, then for any pair
  of consecutive $a_1, a_2$, $a_1 \neq a_2 \wedge a_1' \neq a_2'$.}

\eat{\vera{
2 options:
1) replace - compute the new values (vertex usually), return new graph with only that property. the user can then union back with the original graph to get old and new (this is how GraphX does it). this can uncoalesce.
2) append - compute the new values, add a new property to the bag with that value. this does not uncoalesce. option 2 described above.
}}

