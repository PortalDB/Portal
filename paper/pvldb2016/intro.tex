\section{Introduction}
\label{sec:intro}

% General motivation
The importance of networks in scientific and commercial domains cannot
be overstated.  Networks are represented by graphs, and we will use
the terms ``network'' and ``graph'' interchangeably.  Considerable
research and engineering effort is being devoted to developing
effective and efficient graph representations and analytics.  As of
late, efficient graph abstractions and analytics for {\em static
  graphs} have become available in scope of open source platforms such
as Apache Giraph~\cite{ApacheGiraph}, Apache Spark, through the GraphX
API~\cite{DBLP:conf/osdi/GonzalezXDCFS14}, and GraphLab, through the
PowerGraph library~\cite{DBLP:conf/osdi/GonzalezLGBG12}.  This in turn
makes sophisticated graph analysis methods available to researchers
and practitioners, facilitating their widespread adoption.  Further,
because these systems are open source, this encourages development and
dissemination of new graph analysis methods, and of more efficient
implementations of existing methods.

Arguably the most interesting and important questions one can ask
about networks have to do with their evolution, rather than with their
static state.  Analysis of {\em evolving graphs} has been receiving
increasing attention, with most progress taking place in the last
decade~\cite{DBLP:journals/csur/AggarwalS14,Chan2008,Kan2009,DBLP:journals/tos/MiaoHLWYZPCC15,Ren2011,Semertzidis2015}.
Some areas where evolving graphs are being studied are social network
analysis~\cite{DBLP:conf/kdd/LeskovecBKT08,DBLP:conf/icml/SarkarCJ12},
biological networks~\cite{DBLP:journals/tcsb/BeyerTLSF10} and the Web~\cite{DBLP:journals/jisa/PapadimitriouDG10}.
%\eat{~\cite{DBLP:conf/icwsm/GoetzLMF09,DBLP:journals/tweb/LeskovecAH07,DBLP:conf/kdd/LeskovecBKT08,DBLP:conf/icml/SarkarCJ12}},
%biological networks~\cite{DBLP:journals/tkdd/AsurPU09,DBLP:journals/tcsb/BeyerTLSF10,Stuart2003} and the Web~\cite{DBLP:journals/kais/ChanBL08,DBLP:journals/jisa/PapadimitriouDG10}.
%
Yet, despite much recent interest and activity on the topic, and
despite increased variety and availability of evolving graph data,
{\em systematic support for scalable querying and analytics over
  evolving graphs still lacks}.  This support is urgently needed, due
first and foremost to the scalability and efficiency challenges
inherent in evolving graph analysis, but also to considerations of
usability and ease of dissemination.  {\em In this paper we present
  \ql, a system for scalable exploratory analysis of evolving graphs,
  that fills this gap.}

\ql represents evolution of a graph, including changes in topology and
in attribute values of vertices and edges, continuously through time
using the \tg abstraction.  An example of a \tg is given in
Figures~\ref{fig:tg_rg} and~\ref{fig:tg_ve}, where graph evolution is
shown using two representations.  Figure~\ref{fig:tg_rg} gives the
{\em representative graphs} representation of a \tg --- a sequence of
graphs associated with a sequence of coalesced time periods, and with
the semantics that no change occurred in a graph for the duration of
the time period. Figure~\ref{fig:tg_ve} gives the {\em vertex-edge}
representation of an evolving graph --- a pair of coalesced temporal
SQL relations $V$ and $E$ with appropriate integrity constraints.

\eat{Quick overview of our model and operations.  Point to a
  figure.  We represent graph evolution through a continuous time
  period, graphs have attributes on vertices and edges, vertex and
  edge schemas may evolve.  For graph analysis, it is often useful to
  compute representative graphs - an aggregated view of an evolving
  graph by change or by time period, with quantification.
  Corroborating / complementing information from different sources -
  graph union and intersection.  Graph analytics.}

The \ql system implements the \tg abstraction along with operations of
a fully compositional \tg algebra, motivated by several categories of
questions one may ask about evolving networks as part of exploratory
analysis.  We present some motivating questions next, to give the
flavor of the capabilities of the \ql system.

{\em Which network nodes are showing an increasing popularity trend,
  or have increasing influence, and which are on a downward spiral?}
Node popularity can be quantified by, e.g., node degree, centrality,
or PageRank score.  In the Web graph this information can help
prioritize crawling.  In social networks it can be used for content
recommendation and ad targeting.  In semantic knowledge bases this
information captures the dynamics of zeitgeist.
%
{\em Have any changes in network connectivity been observed, either
  suddenly or gradually over time?} Connectivity can be quantified as,
e.g., pair-wise distance, length of shortest path between communities,
or graph density.  For networks describing insulin-based metabolism
pathways, gradual pathway disruption can be used to determine the
onset of type-2 diabetes~\cite{DBLP:journals/tcsb/BeyerTLSF10}.  For a
website accessibility network, sudden loss of connectivity can signal
that censorship is taking place. In a co-authorship network,
increasing connectivity among topical communities indicates stronger
collaboration across domains.
%
\ql supports efficient computation of node popularity and network
connectivity measures with a combination of {\em temporal aggregation}
and {\em representative graph analytics}.

{\em At what time scale can interesting trends be observed?} The
answer to this question may not be known apriori, at the time when
graph evolution data is being recorded.  For example, changes in node
centrality in a social network may be observable on the scale of
weeks, but not months (coarser).  On the Web, periodic events may
change popularity of websites, with observable trends on the scale of
days, but not hours (finer) or months (coarser).  Furthermore, the
same network may exhibit different trends at different time scales,
e.g., node popularity maychange at a different rate, and thus be
observable on a different scale, than over-all network density.
Understanding at what temporal scale to consider network evolution is
an integral part of exploratory analysis, which \ql supports with {\em
  temporal aggregation}.

{\em Can information about graph evolution be used to make graph
  analytics more stable, or representative?}  Algorithms that compute
website popularly can be vulnerable to link spam, which is a
persistent phenomenon on the Web, but the identity of spammers is
transient~\cite{DBLP:conf/cikm/YangQZGL07}.  This suggests that
persistence vs. transience of a node, edge, or, more generally, of a
subgraph, is a meaningful aspect of quality.  Stable or representative
subgraphs have also been used to improve performance of iterative
computations in evolving graphs, e.g., for computing shortest
paths~\cite{Ren2011}.  The {\em temporal aggregation} operation in \ql
\eat{, available in several interesting variants, }can be used to find
representative subgraphs of an evolving graph.

{\em How can multiple data sources be used jointly, to complement or
  corroborate information about graph evolution?}  It may be the case
that multiple datasets are available, each describing a series of
crawls of different, but possibly overlapping, portions of the Web
graph.  Further, network states may be recorded at different, possibly
overlapping, time periods, or even at different temporal scales.  Can
these datasets be unified, in a principled way, to support analysis or
meta-analysis of network evolution trends?  \ql supports {\em union}
and {\em intersection} operations over evolving graphs that enable
this kind of analysis.

{\bf Contributions.} We make the following contributions to support
efficient and usable analysis of evolving graphs.

\begin{enumerate}[noitemsep,leftmargin=*]
%\begin{description}[noitemsep]

\item We propose a representation of an evolving graph, called a \tg,
  which captures evolution of graph topology, and of attributes of
  vertices and edges, \eat{continuously }through time.  \eat{ Conceptually,
    a \tg corresponds to a pair of
    coalesced~\cite{DBLP:conf/vldb/BohlenSS96} temporal SQL
    relations~\cite{DBLP:journals/sigmod/KulkarniM12} (one describing
    evolution of graph vertices, and the other --- of its edges), with
    appropriate integrity constraints. }The \tg model is presented in
  Section~\ref{sec:model}.

\item We define \tg algebra, which includes temporal and structural
  selection, temporal aggregation with novel quantification
  mechanisms, and \tg intersection and union.  \eat{We show the
    correspondence between \tg algebra operators and temporal SQL when
    appropriate, and describe semantics of operations that depart from
    temporal SQL.}\tg algebra is presented in
  Section~\ref{sec:algebra}.

\item We develop several physical representations of the logical \tg
  data structure, which represent different trade-offs in temporal and
  structural locality.  We implement these \eat{physical }representations
  and the operations of \tg algebra in scope of Apache Spark,
  leveraging the GraphX
  framework~\cite{DBLP:conf/osdi/GonzalezXDCFS14}, and enabling
  distributed computation. System architecture and the implementation
  of the \tg model and algebra are presented in Section~\ref{sec:sys}.

\item An important part of our system is built-in support for
  efficient execution of
  Pregel-style~\cite{DBLP:conf/sigmod/MalewiczABDHLC10} analytics over
  representative graphs of a \tg.  We also support efficient
  computation of trends over attribute values, which may themselves be
  computed, e.g., by applying a graph analytic in the previous step.
  Our support of graph analytics is discussed in
  Section~\ref{sec:analytics}.

\item We conduct an extensive experimental evaluation of the \ql
  system with real datasets.  We demonstrate that good performance can
  be achieved with careful engineering, and that \ql is scalable.  We
  presents results of our evaluation in Section~\ref{sec:exp}, and
  also briefly discuss the usability of our framework in this section.

%\end{description}
\end{enumerate}

\eat{Our language supports a variety of operations including temporal
selection, join, and aggregation, and a rich class of analytics.
Further, we provide a scalable and extensible open-source
implementation of \ql in scope of Apache Spark, an open-source
distributed data processing framework.  We develop several novel
physical representations of evolving graphs, and novel partitioning
strategies that explore the trade-off between structural and temporal
locality.  We experimentally demonstrate that good performance can be
achieved with careful engineering.}

\eat{ {\bf Roadmap.}  We present our model in Section~\ref{sec:model}.
  We describe the \ql query language in Section~\ref{sec:example}, and
  discuss its implementation in an open-source prototype in
  Section~\ref{sec:sys}.  Section~\ref{sec:exp} describes our
  extensive experimental evaluation on real datasets.  We discuss
  related work in Section~\ref{sec:related}.  Future work and
  conclusions are given in Section~\ref{sec:conc}.}
