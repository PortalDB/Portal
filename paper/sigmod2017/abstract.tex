\begin{abstract}

Graphs are used to represent a plethora of phenomena, from the Web and
social networks, to biological pathways, to semantic knowledge
bases. Arguably the most interesting and important questions one can
ask about graphs have to do with their evolution. Which Web pages are
showing an increasing popularity trend? How does influence propagate
in social networks? How does knowledge evolve?  

This paper proposes a logical model of an evolving graph called a \tg,
which captures evolution of graph topology and of its vertex and edge
attributes.  We present a compositional temporal graph algebra \tga,
and show a reduction of \tga to temporal relational algebra with
graph-specific primitives.  We formally study the properties of \tga,
and also show that it is sufficient to concisely express a wide range
of common use cases.

We developed an implementation of our model and algebra in scope of
\ql, built on top of Apache Spark / GraphX.  We developed several
in-memory representations that correspond to different trade-offs in
temporal and structural locality.  An extensive experimental
evaluation on real datasets shows that \ql scales, and that switching
between representations in a multi-operator query can improve
performance.

\end{abstract}
