\begin{abstract}

Graphs are used to represent a plethora of phenomena, from the Web and
social networks, to biological pathways, to semantic knowledge
bases. Arguably the most interesting and important questions one can
ask about graphs have to do with their evolution. Which Web pages are
showing an increasing popularity trend? How does influence propagate
in social networks? How does knowledge evolve?

\eat{Much research and engineering effort today goes into developing
sophisticated graph analytics and their efficient implementations,
both stand-alone and in scope of data processing platforms. Yet, {\em
  systematic support} for scalable querying and analytics over {\em
  evolving graphs} still lacks.}

In this paper we address the need to enable {\em systematic support}
for scalable querying and analytics over {\em evolving graphs}.  We
propose a representation of an evolving graph, called a \tg, which
captures evolution of graph topology, and of attributes of vertices
and edges, continuously through time.  We develop a compositional \tg
algebra that includes such operations as temporal selection, subgraph,
aggregation, and a rich class of analytics.  We present \ql, a system
that implements our model and algebra in scope of Apache Spark, an
open-source distributed data processing framework.  We develop
multiple physical representations of evolving graphs and study the
trade-offs between structural and temporal locality.  We provide an
extensive experimental evaluation on real datasets, demonstrating that
careful engineering can lead to good performance.

\end{abstract}
