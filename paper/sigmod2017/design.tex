\subsection{Design choices}
\label{sec:algebra:prelim}

Our goal in developing \tga is to give users an ability to concisely
express a wide range of common analysis tasks, while at the same time
preserving inter-operability with standard SQL.  For this reason our
data model is based on the temporal relational model, and our algebra
corresponds to temporal relational calculus, a two-sorted logic with
variables and quantifiers explicitly ranging over the time
domain~\cite{DBLP:reference/db/Toman09}, for graphs.  We will make
this relationship precise in Section~\ref{sec:formal}.

As a consequence, our data structure and algebraic operations can be
implemented on top of a relational engine.  Implementing
(non-temporal) graph querying and analytics in an RDBMS has been
receiving renewed
attention~\cite{DBLP:conf/sigmod/AbergerTOR16,DBLP:conf/sigmod/SunFSKHX15,DBLP:journals/pvldb/Xirogiannopoulos15},
and our work is in-line with this trend.

\tga does not support general recursion or transitive closure
computation. (Although, as we will see in Section~\ref{sec:analytics},
Pregel-style graph analytics such as PageRank are supported as an
extension.)  For this reason, we also do not support regular path
queries (RPQ) or the more general path query classes (CRPQ and ECRPQ).
Extending our formalism with recursion and path queries is
non-trivial, and we leave this to future work, see
Section~\ref{sec:conc} for a discussion.

Rather than focusing on path computation and graph traversal, we
stress tasks that perform whole-graph analysis over time.  Several
such tasks were described in Section~\ref{sec:cases}.  Additional
examples can be found in SocialScope~\cite{Amer-Yahia2009} --- a
closed non-temporal algebra for multigraphs that is motivated by
information discovery in social content sites.  It is not difficult to
show that the graph (rather than multi-graph) versions of all
SocialScope operations can be expressed, and augmented with the
temporal dimension, in \tga.

