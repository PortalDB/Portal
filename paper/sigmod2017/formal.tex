\section{Expressive power}
\label{sec:formal}

In this section we study expressiveness of the \tg model, which
consists of the \tg data structure (Definition~\ref{def:tg}) and of
\tg algebra for querying the data structure
(Sections~\ref{sec:algebra:unary}-~\ref{sec:algebra:composition}).  We
stress that ours is a {\em valid-time data model} that does not have
transaction-time and bitemporal support.  We start by proposing two
natural notions of completeness for a temporal graph query language.

\begin{definition}
  Let $L$ be a temporal relational language and $\tve$ --- a
  relational vertex-edge representation of a temporal graph.  An
  \edgeq $q^t_e$ in $L$ takes a graph $\tve (\tv, \te, \tav, \tae)$ as
  input, and outputs another graph $\tve'$ on the vertices of $\tve$
  such that the edges of $\tve'$ are defined by $q_e$.  A language is
  $L$-\edgec if it can express each $q_e$ in $L$.
  \label{def:edgecomplete}
\end{definition}

Note that the query $q_e$ is not restricted to act on \te alone, and
may refer to the other constituent relations \tve.

\julia{Serge, in Definition~\ref{def:edgecomplete}, does it matter
  whether $q^t_e$ can use additional non-graph relations in the
  schema (temporal or not), or does expressiveness stay the same?}

\begin{definition}
  Let $L$ be a temporal relational language, and let $\tve$ be a
  relational vertex-edge representation of a temporal graph.  An
  \vertexq $q^t_v$ in $L$ takes a graph $\tve (\tv, \te)$ as input,
  and outputs another graph $\tve'$ such that the vertices of $\tve'$
  are defined by $q_v$. A language is $L$-\vertexc if it can express
  each $q_v$ in $L$.
\label{def:vertexcomplete}
\end{definition}

We now refer to definitions~\ref{def:edgecomplete}
and~\ref{def:vertexcomplete} and show that \ql algebra is \edgec and
\vertexc, with respect to the valid-time fragment of temporal
relational algebra (TRA).  TRA is an algebra that corresponds to
temporal relational calculus
(TRC)~\cite{DBLP:reference/db/ChomickiT09b}, a first-order logic that
extends relational calculus and supports variables and quantifiers
over both the data domain and time domain.

\begin{theorem}
\ql algebra is TRA-\edgec.
\label{th:edgecomplete}
\end{theorem}

\begin{proof}  Forthcoming.
\end{proof}

\begin{theorem}
\ql algebra is TRA-\vertexc.
\label{th:vertexcomplete}
\end{theorem}

\begin{proof}  Forthcoming.
\end{proof}

We showed in Theorems~\ref{th:edgecomplete}
and~\ref{th:vertexcomplete} that \ql can compute vertices and edges of
a \tg based on arbitrary TRA queries.  We also showed in
Section~\ref{sec:algebra} that every \ql operator can be implemented
by a sequence of TRA queries.

% S-reducibility and exteded S-reducibility
Consider a non-temporal query langauge $L$ and its temporal extension
$L^t$.  For a point-based model, it is customary to interrogate two
properties --- snapshot reducibility (S-reducibility) and extended
snapshot reducibility (extended S-reducibility).

S-reducibility states that for every query $q$ in $L$, there must
exist a syntactically similar query $q^t$ in $L^t$ that generalizes
$q$.  Specifically the following relationship should hold when $q^t$
is evaluated over a temporal database $D^t$ (recall that $\tau$ is the
temporal slice operator): $q(\tau_c(D^t)) = \tau_c(q^t(D^t))$, for all
time points $c$.  Extended S-reducibility requires that $L^t$ provide
an ability to make explicit references to timestamps alongside
non-temporal predicates.

We showed above that \ql algebra is TRC-\edgec and TRC-\vertexc.
Since TRC is known to be S-reducible and extended S-reducible with
respect to (non-temporal) relational calculus, then these properties
also hold over the \edgeq and \vertexq of \ql.

% Make a connection to sequenced semantics - we don't support change
% preservation.


