%\section{Lazy Coalescing}
\subsection{Lazy Coalescing}
\label{sec:sys:coal}

\eat{
\begin{table}
\small
\begin{tabular}{ l | p{4cm} | c }
\hline
\multicolumn{1}{l|}{\bfseries Operation} & \multicolumn{1}{c|}{\bfseries Requires Coalesced Input} & \multicolumn{1}{c}{\bfseries Uncoalesces} \\ \hline
slice & no & no \\ \hline
select & only if predicate incl. interval & no \\ \hline
project & no & yes \\ \hline
aggregate & only if structural group by is used and includes interval & yes \\ \hline
union & no & yes \\ \hline
intersection & no & yes \\ \hline
analytics & no & yes \\
\hline
\end{tabular}
\caption{Coalesce analysis.}
\label{tab:coalesce}
\end{table}
}

All operations and all sequences of operations must output a valid
temporally coalesced \tg.  Several implementations are possible for
the coalesce operation over temporal SQL relations,
see~\cite{DBLP:conf/vldb/BohlenSS96} for details.  We use the
partitioning method, where the relation (e.g., \tv, \te, \trg) is
grouped by key, and tuples are sorted and folded within each group to
produce time periods of maximum length.  This involves shuffling
between partitions, is computationally expensive, and motivates
lazy coalescing.

Correctness of many \tg algebra operations does not depend on whether
their input is coalesced.  Consequently, \ql supports both eager and
lazy coalescing, with lazy being the default for queries that admit
it.  We now present useful coalescing rules (\ttt stands for either
\trg or \tve).

\eat{ As with duplicate elimination, if the coalesced output is
  expected to be significantly reduced in size, such as after a
  project operation on a high volatility attribute, performing
  coalesce eagerly before a time-consuming operation, especially
  analytics, can be advantageous.}
%

\eat{{\bf Successive coalescing is unnecessary} in \tg algebra just as it
is for (non-graph) temporal
databases~\cite{DBLP:conf/vldb/BohlenSS96}:}
\eat{\begin{equation}
\cl(\cl(\ttt)) \equiv \cl(\ttt)
\label{rul:twice}
\end{equation}}

\eat{It is unnecessary to coalesce an already coalesced \ttt, e.g., if the
data on disk is known to be coalesced:}

\eat{\begin{equation} 
\cl(\ttt) \equiv \ttt~~\text{iff}~\textsf{is\_coalesced}(\ttt)
\label{rul:cond}
\end{equation}}

%The first rule eliminates successive coalescing:

%\begin{equation}
%\cl(\cl(\ttt)) \equiv \cl(\ttt)
%\label{rul:coalr0}
%\end{equation}

{\bf Slice allows lazy coalescing.}  We showed in
Section~\ref{sec:algebra:slice} that slice does not destroy
coalescing.  Further, recall that $\tau_c(\cl(\ttt))$ returns tuples
$(x,p \cap c)$, for which $p \cap c \neq \emptyset$.  If \ttt is
uncoalesced, then there exist some value-equivalent tuples $(x, p_1)$
and $(x, p_2)$
s.t. $\pred{p_1}{meets}{p_2}~\lor~\pred{p_1}{contains}{p_2}~\lor$\\$\pred{p_1}{overlaps}{p_2}$,
which would be replaced by $(x, p_1 \cup p_2)$ in \cl(\ttt).  Because
intersection distributes over union, i.e.,$(p_1 \cup p_2) \cap c =
(p_1 \cap c) \cup (p_2 \cap c)$, coalescing can be equivalently
performed before or after slice: \eat{$\cl(\tau_c(\ttt)) \equiv
\tau_c(\cl(\ttt))$.}
\begin{equation}
\cl(\tau_c(\ttt)) \equiv \tau_c(\cl(\ttt))
\label{rul:slice}
\end{equation}

{\bf Temporal subgraph allows lazy coalescing.} It was shown
in~\cite{DBLP:conf/vldb/BohlenSS96} that coalescing can be deferred
until after selection if the selection condition is independent of the
valid time of the input. Since temporal subgraph
(Section~\ref{sec:algebra:subgraph}) is a time-invariant form of
select, coalescing can be deferred.  Subgraph does not uncoalesce \tve
but may uncoalesce \trg (Section~\ref{sec:algebra:subgraph}). For this
reason it is required to coalesce the final output in
Rule~\ref{rul:subgraphrg} even if the input was eagerly coalesced, but
this is not required in Rule~\ref{rul:subgraphve}.
\begin{equation}
\cl(\sigma_{C_V,C_E}(\tve)) \equiv \sigma_{C_V,C_E}(\cl(\tve))
\label{rul:subgraphve}
\end{equation}
\begin{equation}
\cl(\sigma_{C_V,C_E}(\trg)) \equiv \cl(\sigma_{C_V,C_E}(\cl(\trg)))
\label{rul:subgraphrg}
\end{equation}

Deferring coalescing can be quite effective for
Rules~\ref{rul:slice},~\ref{rul:subgraphve} and~\ref{rul:subgraphrg}
if the selectivity of the predicate is high.

{\bf Temporal map allows lazy coalescing}, since it processes each
input tuple without modifying the corresponding time intervals, and
independently of the valid time.  Map destroys coalescing
(Section~\ref{sec:algebra:project}), and requires to coalesce the
output even if the input was eagerly coalesced.
\begin{equation}
\cl(\map_{M_V,M_E}(\ttt)) \equiv \cl(\map_{M_V,M_E}(\cl(\ttt)))
\label{rul:map}
\end{equation}

\eat{Rules in ~\ref{rul:coalr1,rul:coalr2,rul:coalr3} exploit the fact that
some operations preserve coalescing, as shown in
Section~\ref{sec:algebra}:}

\eat{\begin{equation}
coal(\tau_c(coal(T))) \equiv \tau_c(coal(T))
\label{rul:coalr1}
\end{equation}
\begin{equation}
coal(\sigma_f(coal(\tve))) \equiv \sigma_f(coal(\tve))
\label{rul:coalr2}
\end{equation}
\begin{equation}
coal(udf(coal(T))) \equiv udf(coal(T))
\label{rul:coalr3}
\end{equation}}

\eat{Rules in
~\cref{rul:coalr4,rul:coalr5,rul:coalr6,rul:coalr7,rul:coalr8,rul:coalr9}
cover operations which are invariant with respect to time and so
coalescing can be postponed until after the operation:}

\eat{\begin{equation}
\tau_c(coal(T)) \equiv coal(\tau_c(T))
\label{rul:coalr4}
\end{equation}}

\eat{Recall that slice selects each entity with an interval that intersects
with $c$ and modifies only the selected tuple periods to $p \cap c$.
According to the rule~\ref{rul:coalr1}, slice does not uncoalesce.  If
the input relation is uncoalesced, it contains some value-equivalent
tuples with adjacent or overlapping periods.  According to the
definitions of $intersect$ and $coalesce$, an intersect with interval
$p$ is the same as the coalesce of intersects of intervals composing
$p$.  I.e., $p \cap c = (p1 \cap c) \cup (p2 \cap c$) where $p = p1
\cup p2$.  Thus coalescing can be performed before or after slice.}

\eat{\begin{equation}
coal(\sigma_f(coal(T))) \equiv coal(\sigma_f(T))
\label{rul:coalr5}
\end{equation}}

\eat{As shown in~\cite{DBLP:conf/vldb/BohlenSS96}, coalescing can be
deferred until after the selection if the selection condition is
independent of the valid time of the input. Since subgraph is a
time-invariant form of select, coalesce is not required prior to
subgraph.  For both this rule and the slice rule above, deferring
coalescing can be quite effective if the selectivity of the predicate
is high.}

\eat{\begin{equation}
coal(\mu(coal(T))) \equiv coal(\mu(T))
\label{rul:coalr6}
\end{equation}}

\eat{The map operation processes each tuple without modifying the
corresponding intervals and independently of the valid time of the
input.  Thus, as above, coalesce can be deferred.}

\eat{\begin{equation}
coal(coal(T_1) \cap coal(T_2)) \equiv coal(T_1 \cap T_2)
\label{rul:coalr7}
\end{equation}}

\eat{\begin{equation}
coal(coal(T_1) \cup coal(T_2)) \equiv coal(T_1 \cup T_2)
\label{rul:coalr8}
\end{equation}}

\eat{For \trg, temporal graph intersection is a inner theta-join followed
by projection.  As shown in~\cite{DBLP:conf/vldb/BohlenSS96},
coalescing can be deferred until after a join and after a projection
if that projection is independent of the valid time interval of the
tuple, which is the case here.  Thus we can defer coalescing until
after the temporal graph intersection.  Temporal graph union is also a
join, albeit an outer one, so the same argument applies.}

\eat{\begin{equation}
coal(udf(coal(T))) \equiv coal(udf(T))
\label{rul:coalr9}
\end{equation}}

{\bf Temporal aggregation requires eager coalescing.}  When
aggregating by change, the input must be coalesced to compute correct
aggregation windows.  If the input is not coalesced, then there may be
two or more consecutive intervals which should be treated as one but
would count separately. For aggregation by time, some aggregate
functions also require coalesced input.  For example, \insql{count}
will produce different result if several consecutive or overlapping
value-equivalent tuples fall within the same window.  Thus we add the
following conditional coalescing rule:
\begin{fleqn}[0pt]
\begin{multline}
\cl(\gamma_{W,Q_V,Q_E,A_V,A_E}(\cl(\ttt))) \equiv \cl(\gamma_{W,Q_V,Q_E,A_V,A_E}(\ttt)) \\
\text{iff}~~W \text{by\_time} \wedge A_V, A_E \in \{\insql{any},\insql{first}, \insql{last},\insql{min}, \insql{max} \}
\label{rul:agg}
\end{multline}
\end{fleqn}

\insql{any} returns any one of the values associated with a property
within each group in $W$, and so is invariant to time and to the
number of tuples in $W$.  \insql{first} and \insql{last} pick the
earliest (resp. latest) value within each group in $W$.  If \ttt is
uncoalesced, there is at least one pair of value-equivalent tuples
with overlapping or consecutive periods.  Selecting between equivalent
values will generate the same result.  The same argument applies to
\insql{min} and \insql{max}.

{\bf Temporal intersection and union allow lazy coalescing.} For \trg,
temporal graph intersection (Section~\ref{sec:algebra:join}) is a join
followed by projection.  As shown in~\cite{DBLP:conf/vldb/BohlenSS96},
coalescing can be deferred until after a join, and after a projection
if projection is independent of the valid time interval of the tuple,
which is the case here.  Thus we can defer coalescing until after
temporal graph intersection.  Temporal graph union
(Section~\ref{sec:algebra:outerjoin}) is an outer join, and the same
argument applies.  Note that both operations may destroy coalescing,
and so a final coalesce is required over the result even if inputs
were eagerly coalesced.
\begin{equation}
\cl(\cl(\ttt_1) \cap \cl(\ttt_2)) \equiv \cl(\ttt_1 \cap \ttt_2)
\label{rul:intersect}
\end{equation}
\begin{equation}
\cl(\cl(\ttt_1) \cup \cl(\ttt_2)) \equiv \cl(\ttt_1 \cup \ttt_2)
\label{rul:union}
\end{equation}

{\bf Temporal user-defined analytics allow lazy coalescing.} If \trg
is not coalesced, it contains one or more value-equivalent
representative graphs with consecutive or overlapping periods.
Analytics are time-invariant functions applied to each representative
graph.  Thus when applied to value-equivalent graphs, they will
produce equivalent results, and so coalesce can be deferred.  Further,
analytics to not destroy coalescing, and so it is not required to
coalesce again over the final result.  
\begin{equation}
\cl(\uda(\ttt)) \equiv \uda(\cl(\ttt))
\label{rul:analytic}
\end{equation}
Although it is possible to defer coalesce for analytics, this will not
be done in practice.  Analytics are computationally expensive, and it
is usually beneficial to eagerly coalesce the input, reducing the
number of representative graphs.

%Finally, a rule that eliminates coalescing of coalesced \tgs:

\eat{\begin{equation}
coal(T) \equiv T~~iff~~is\_coalesced(T)
\label{rul:coalr11}
\end{equation}}

%This last rule is very useful if the data on disk is known to be
%coalesced.

