\appendix 
\section{Integrity proofs}
\label{sec:app1}

B\"ohlen et al.~\cite{DBLP:conf/vldb/BohlenSS96} show that temporal
selection, Cartesian product and difference all produce a coalesced
relation as output if the input was coalesced.  They also show that
temporal union and temporal projection can give rise to an uncoalesced
output even if the inputs were coalesced.  Intuitively, this is
because union and projection can give rise to duplicates in
traditional relational algebra, and lack of coalescing is the temporal
analogy to a duplicate.
%
These observations also hold in our scenario at the level of
individual relations, each containing representative graphs, vertices,
edges, or vertex/edge attributes.

{\bf Slice does not uncoalesce.} Whether evaluated over \trg or \tve,
slice is guaranteed to return a coalesced relation when evaluated over
a coalesced input.  This is because, for any relation \insql{R}, there
will be at most one tuple from \insql{R} in the result of $\tau_c (R)$,
with a validity period that is either the same as it was in \insql{R},
or further restricted (trimmed).  Therefore, there is no need to
coalesce \trg after slice, or on lines 1-4 of Algorithm~\ref{alg:op}
when operating over \tve.

{\bf Slice does not require FK enforcement for \tve.}  To see why,
consider an edge $\te(v_1, v_2, p)$ and one of the corresponding
vertices $\tv(v_1, p_1)$, such that $\pred{p_1}{contains}{p}$ (per
Definition~\ref{def:tg} condition~\ref{def:tg:c1}).  Suppose now that
slice was applied to \tv and to \te with condition $c$.  Is it
possible that edge $(v_1, v_2, p \cap c)$ is in the result of $\tau_c
(\te)$ (i.e., $p \cap c \neq \emptyset$), while vertex $\tv(v_1, p_1 \cap
c)$ is not in the result of $\tau_c (\tv)$ (i.e., $p_1 \cap c =
\emptyset$)?  Clearly, the answer is no, since
$\pred{p_1}{contains}{p}$, and so it must be the case that
$\pred{p_1}{contains}{(p \cap c)}$.  A similar argument justifies that
FK enforcement is not needed for $\tau_c(\tav)$ (w.r.t. $\tau_c (\tv)$) and
for $\tau_c(\tae)$ (w.r.t. $\tau_c (\te)$).

\eat{{\bf Subgraph may uncoalesce \trg.} Consider \insql{T1} in
Figure~\ref{fig:tg_rg}.  The query
$\sigma_{C_V:{school='Drexel'},C_E:\top} (\insql{T1})$ matches
vertices $v_1$ and $v_3$ in every representative graph in which these
occur.  Since graphs for time periods $[1/15,2/15)$ through
  $[6/15,7/15)$ are identical, the result will be uncoalesced, and
    will need to be coalesced explicitly.  The final result will
    consist of 2 representative graphs, with both $v_1$ and $v_3$ for
    $[1/15, 7/15)$, and with $v_3$ only for $[7/15, 10/15)$.
}

{\bf Subgraph does not uncoalesce \tve.}  Consider again the
computation of $\tv'$ described above, with a query that involves
projection, selection and join over temporal SQL relations $\tv$ and
\tav.  While selection and join cannot produce an uncoalesced output
if the input is coalesced, projection may produce an uncoalesced
output relation~\cite{DBLP:conf/vldb/BohlenSS96}.  Interestingly,
projection does not result in an uncoalesced output in this case. To
see why, suppose that $C_V$ is trivial, i.e., that $\sigma_{C_{V1}}
(\tv) = \tv$ and $\sigma_{Q_{C2}} (\tav) = \tav$. Then $\tv' =
\pi_{v,p} (\tv \bowtie \tav)$, and since $\tv \bowtie \tav$ is a
primary key-foreign key join, then $\tv' = \tv$.  If $C_V$ is
non-trivial, i.e., $\sigma_{Q_{C1}} (\tv) \subset \tv$ or
$\sigma_{C_{V2}} (\tav) \subset \tav$, then it will be the case that
$V' \subset V$.  In both cases, if $\tv$ is coalesced then so is
$\tv'$.  A similar argument applies to the edges relation $\te'$.
Finally, since $\tav'$ and $\tae'$ are computed from coalesced input
relations using only selection, they are guaranteed to be coalesced.
Thus, it is not necessary to coalesce on lines 1-4 of
Algorithm~\ref{alg:op}.

{\bf Subgraph requires FK enforcement for \tve.}  A natural query
specifies a selection condition over the vertices, and computes the
vertex-induced subgraph.  In this case we cannot compute $\te'$ from
$\te$ alone, but will need to remove edges for which one or both
vertices are not present in $\tv'$.  Similarly, we must remove tuples
from $\tav'$ and $\tae'$ for which no corresponding tuples exist in
$\tv'$ and $\te'$.

{\bf Map may uncoalesce \tav, \tae and \trg.}  Consider computing
$\map_{m_V:name, M_E:\emptyset} (\insql{T1})$ over \insql{T1} in
Figure~\ref{fig:tg_ve}.  There will be three identical tuples in the
result for vertex $v_2$ for $[2/15, 5/15)$, $[5/15, 7/15)$ and $[7/15,
      10/15)$, which must be coalesced to return a valid \tav.  A
      similar argument holds for \tae. This operation will also
      produce two identical representative graphs in \trg in
      Figure~\ref{fig:tg_rg} for $[2/15, 5/15)$ and $[5/15, 6/15)$,
          which will have to be coalesced explicitly.

{\bf Map does not require FK enforcement for \tve.}  This is because
only \tav and \tae are affected, while the contents of \tv and \te
remain as in the input.

{\bf Intersection may uncoalesce \tve and \trg.}  While temporal join over
coalesced temporal relations does not uncoalesce, as shown
in~\cite{DBLP:conf/vldb/BohlenSS96}, it is followed by a projection,
which does uncoalesce.  For example, consider a simple \tg $T_2$
that consists of only 1 vertex $v_1$ with attribute $(name:Alice)$ and
no edges and exists for the period $[1/14, 5/15)$.  The
  intersection of $T_2$ with \insql{T} in Figure~\ref{fig:tg_rg}
  produces the same graph with vertex $v_1$ for each period of
  intersection $[1/15, 2/15), [2/15, 5/15)$, which must be coalesced.

{\bf Intersection requires FK enforcement for \tav and \tae but not
  for \te.}  \eat{The proof for this is similar to the one for
  \insql{slice}. } Consider an edge $\te(v_1, v_2, p^e) \in \te'$ and
one of the corresponding vertices $\tv(v_1, p^v) \in \tv'$ computed in
Algorithm~\ref{alg:inter}.  To violate the FK constraint on \te',
there must exist a time instant $t \in p^e \wedge t \not\in p^v$.  But
that is not possible, since by definition of temporal join $p^e$
exists in both $\te_1$ and $\te_2$, and $\tve_1$ and $\tve_2$ are
valid \tgs.  To see why we need to enforce FK for $\tav'$, consider
again the example above with $T \cap T_2$.  The temporal outer join of
$\tav_1$ and $\tav_2$ will produce a tuple for period [1/14, 1/15)
  with the attribute $(name:Alice)$ from $\tav_2$.  Clearly this tuple
  violates the FK constraint because we do not have any vertices in
  $\tv'$ prior to 1/15.  Similarly for $\tae'$.

{\bf Union may uncoalesce \tve and \trg.}  The logic here is similar
to the case with temporal intersection.  Because we produce a single
graph for each time period by computing a union of $g_1$ and $g_2$, a
form of projection, the result may be uncoalesced.  For example,
consider a simple \tg $T_2$ that consists of a single graph equal to
the first representative graph of \insql{T} in Figure~\ref{fig:tg_rg}
valid for a period of $[12/14, 5/15)$.  The same representative graph
  will be produced for period $[12/14, 1/15)$ and $[1/15, 2/15)$ and
      must be coalesced.  $\tv'$ and $\te'$ are coalesced since they
      are computed with a temporal
      join~\cite{DBLP:conf/vldb/BohlenSS96}.  However, $\tav'$ and
      $\tae'$ are computed with a join followed by a projection, which
      does uncoalesce.  Thus we apply the coalescing operation on
      lines 3 and 4.

{\bf Union does not require FK enforcement for \tve.} Temporal full
outer join produces a vertex tuple for all periods in $\tv_1$ and
$\tv_2$.  If $\tve_1$ and $\tve_2$ are valid graphs, then 
referential integrity holds for $\te'$, $\tav'$ and $\tae'$.

{\bf Difference may uncoalesce \tve and \trg.}  The logic here is
similar to the temporal union and intersection cases above.  Because
we produce a single graph for each time period by computing a $g_1
\setminus g_2$, a form of projection, the result may be uncoalesced.
For example, consider a simple \tg $T_2$ as above, i.e. equal to the
first representative graph of \insql{T} in Figure~\ref{fig:tg_rg}
valid for a period of $[2/15, 7/15)$.  The same representative graph
  will be produced for period $[5/15, 6/15)$ and $[6/15, 7/15)$ and
      must be coalesced.  As with union and intersection, $\tv'$ and
      $\te'$ are coalesced since they are computed with a temporal
      join~\cite{DBLP:conf/vldb/BohlenSS96}.  However, $\tav'$ and
      $\tae'$ are computed with a join followed by a projection, which
      does uncoalesce.  Thus we apply the coalescing operation on
      lines 3 and 4.

{\bf Difference requires FK enforcement for \tav and \tae but not \te.}
The proof for this is the same as for intersection and we will not
repeat it.

{\bf Node creation may uncoalesce \tve and \trg.}  Consider
\insql{T} that consists of only 1 vertex $v_1$ and no edges, and
suppose that $v_1$ exists during the odd months of the year.  \tv
contains $(v_1, [1/16, 2/16)), \ldots,$ $(v_1, [11/16,12/16))$, for
    which time periods do not meet and so these tuples cannot be
    coalesced.  Next consider $\tve'$ in the result of
    $create_{W=2~months, Q_V=exists,Q_E=all}(T)$.  This relation will
    contain $(v_1, [1/16, 3/16)), \ldots,$ $,(v_1, [11/16,1/17))$ ---
        one tuple per input tuple, all tuples meet and so \tv' must be
        coalesced.  A similar argument holds for, \te, \tav, \tae,
        \trg.

{\bf Node creation requires FK enforcement for \tve.}  As
illustrated in Figure~\ref{fig:tg_agg1}, $(v_1,v_2,[1/15,4/15))$ is
  absent from $\te'$ because, although this edge was present in \tv
  for part of the period, and so meets the condition
  $Q_E=\insql{exists}$, vertex $(v_2,[1/15,5/15))$ is absent from
    $\tv'$.

We now give an analysis of whether FK enforcement is required, by
considering the relationship between vertex and edge aggregation
quantifiers $Q_V$ and $Q_E$.  Recall that we support quantifiers
\insql{all}, \insql{most}, \insql{at least} $n$, and \insql{exists},
and observe that they can be translated to a threshold on the
percentage of the time during which an entity (vertex or edge)
existed, relative to the duration of the window specification: $t = 1$
for \insql{all}, $t > 0.5$ for \insql{most}, $t > 0$ for
\insql{exists} and $t > n$ for \insql{at least} $n$.  Let us refer to
the vertex threshold as $t_V$, and to the edge threshold as $t_E$.  If
an entity's existence meets the threshold, it will be retained in the
result.

FK enforcement is only required if $t_V > t_E$.  To see why, consider
an interval $w$ which is one of the windows computed based on
specification $W$.  We produce an edge $(v_1, v_2, w)$ iff $\exists
\{(v_1, v_2, p_1),\ldots,(v_1, v_2, p_k)\} \in
\te~~|~~(\bigcup_{i=1}^k p_i \cap w) / w > t_E$.  According to
Definition~\ref{def:tg:c1}, if the input graph is valid then we must
also produce $(v_1, w)$ and $(v_2, w)$ if $t_V \leq t_E$, since vertex
periods are supersets of edge periods.  However, if $t_V > t_E$, we
may not produce a vertex that meets $t_E$ but does not meet the
greater $t_V$, as we showed with the example above.

\eat{FK enforcement is only required if $t_V$ is stricter
than $t_E$.  To see why, consider 2 cases.  (1) Suppose that $t_V \leq
t_E$.  To violate FK constraint, we must produce edge $(v_1, v_2, p)
\in \te'$ and not produce vertex $(v_1, p) \in \tv'$.  This edge
$(v_1, v_2)$ must have been valid during at least $t_E$ portion of p
in \te.  If our input is valid, then there must exist both $v_1$ and
$v_2$ in \tv during at least $t_E$ portion of p to pass the FK
constraint.  Since $t_E$ is at least as large as $t_V$, we will
produce both $(v_1, p)$ and $(v_2, p)$.  (2) $t_V > t_E$.  The proof
here is similar to above.  If we produce edge $(v_1, v_2, p) \in
\te'$, then there must be both $v_1$ and $v_2$ in \tv that exist
during at least $t_V$ portion of p.  However, since $t_V$ is larger
than $t_E$, we may not produce a vertex in \tv' that meets $t_E$ but
does not meet the greater $t_V$.  In that case the edge is violating
the FK constraint.}

{\bf Analytics do not uncoalesce.} An analytic adds a new key-value
pair to the property bag of each vertex $v$ of every representative
graph $g$, and does not implicitly project out vertex attributes.
\eat{For this reason}Thus, an analytic cannot cause two temporally
adjacent graphs \eat{$g_1$ and $g_2$ }that were not equivalent before
the attribute addition \eat{new attribute was added }to become
equivalent after it was added.

\section{Additional results.}
\label{sec:app2}

Plots and discussion in this section complement experimental results
presented in Section~\ref{sec:exp}.

