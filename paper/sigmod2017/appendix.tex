\appendix 
\section{Integrity proofs}
\label{sec:app1}

B\"ohlen et al.~\cite{DBLP:conf/vldb/BohlenSS96} show that temporal
selection, Cartesian product and difference all produce a coalesced
relation as output if the input was coalesced.  They also show that
temporal union and temporal projection can give rise to an uncoalesced
output even if the inputs were coalesced.  Intuitively, this is
because union and projection can give rise to duplicates in
traditional relational algebra, and lack of coalescing is the temporal
analogy to a duplicate.
%
These observations also hold in our scenario at the level of
individual relations, each containing representative graphs, vertices,
edges, or vertex/edge attributes.

{\bf Slice does not uncoalesce.} Whether evaluated over \trg or \tve,
slice is guaranteed to return a coalesced relation when evaluated over
a coalesced input.  This is because, for any relation \insql{R}, there
will be at most one tuple from \insql{R} in the result of $\tau_c (R)$,
with a validity period that is either the same as it was in \insql{R},
or further restricted (trimmed).  Therefore, there is no need to
coalesce \trg after slice, or on lines 1-4 of Algorithm~\ref{alg:op}
when operating over \tve.

{\bf Slice does not require FK enforcement for \tve.}  To see why,
consider an edge $\te(v_1, v_2, p)$ and one of the corresponding
vertices $\tv(v_1, p_1)$, such that $\pred{p_1}{contains}{p}$ (per
Definition~\ref{def:tg} condition~\ref{def:tg:c1}).  Suppose now that
slice was applied to \tv and to \te with condition $c$.  Is it
possible that edge $(v_1, v_2, p \cap c)$ is in the result of $\tau_c
(\te)$ (i.e., $p \cap c \neq \emptyset$), while vertex $\tv(v_1, p_1 \cap
c)$ is not in the result of $\tau_c (\tv)$ (i.e., $p_1 \cap c =
\emptyset$)?  Clearly, the answer is no, since
$\pred{p_1}{contains}{p}$, and so it must be the case that
$\pred{p_1}{contains}{(p \cap c)}$.  A similar argument justifies that
FK enforcement is not needed for $\tau_c(\tav)$ (w.r.t. $\tau_c (\tv)$) and
for $\tau_c(\tae)$ (w.r.t. $\tau_c (\te)$).

{\bf Subgraph may uncoalesce \trg.} Consider \insql{T1} in
Figure~\ref{fig:tg_rg}.  The query
$\sigma_{C_V:{school='Drexel'},C_E:\top} (\insql{T1})$ matches
vertices $v_1$ and $v_3$ in every representative graph in which these
occur.  Since graphs for time periods $[1/15,2/15)$ through
  $[6/15,7/15)$ are identical, the result will be uncoalesced, and
    will need to be coalesced explicitly.  The final result will
    consist of 2 representative graphs, with both $v_1$ and $v_3$ for
    $[1/15, 7/15)$, and with $v_3$ only for $[7/15, 10/15)$.

{\bf Subgraph does not uncoalesce \tve.}  Consider again the
computation of $\tv'$ described above, with a query that involves
projection, selection and join over temporal SQL relations $\tv$ and
\tav.  While selection and join cannot produce an uncoalesced output
if the input is coalesced, projection may produce an uncoalesced
output relation~\cite{DBLP:conf/vldb/BohlenSS96}.  Interestingly,
projection does not result in an uncoalesced output in this case. To
see why, suppose that $C_V$ is trivial, i.e., that $\sigma_{C_{V1}}
(\tv) = \tv$ and $\sigma_{Q_{C2}} (\tav) = \tav$. Then $\tv' =
\pi_{v,p} (\tv \bowtie \tav)$, and since $\tv \bowtie \tav$ is a
primary key-foreign key join, then $\tv' = \tv$.  If $C_V$ is
non-trivial, i.e., $\sigma_{Q_{C1}} (\tv) \subset \tv$ or
$\sigma_{C_{V2}} (\tav) \subset \tav$, then it will be the case that
$V' \subset V$.  In both cases, if $\tv$ is coalesced then so is
$\tv'$.  A similar argument applies to the edges relation $\te'$.
Finally, since $\tav'$ and $\tae'$ are computed from coalesced input
relations using only selection, they are guaranteed to be coalesced.
Thus, it is not necessary to coalesce on lines 1-4 of
Algorithm~\ref{alg:op}.

{\bf Subgraph requires FK enforcement for \tve.}  A natural query
specifies a selection condition over the vertices, and computes the
vertex-induced subgraph.  In this case we cannot compute $\te'$ from
$\te$ alone, but will need to remove edges for which one or both
vertices are not present in $\tv'$.  Similarly, we must remove tuples
from $\tav'$ and $\tae'$ for which no corresponding tuples exist in
$\tv'$ and $\te'$.

{\bf Map may uncoalesce \tav, \tae and \trg.}  Consider computing
$\map_{m_V:name, M_E:\emptyset} (\insql{T1})$ over \insql{T1} in
Figure~\ref{fig:tg_ve}.  There will be three identical tuples in the
result for vertex $v_2$ for $[2/15, 5/15)$, $[5/15, 7/15)$ and $[7/15,
      10/15)$, which must be coalesced to return a valid \tav.  A
      similar argument holds for \tae. This operation will also
      produce two identical representative graphs in \trg in
      Figure~\ref{fig:tg_rg} for $[2/15, 5/15)$ and $[5/15, 6/15)$,
          which will have to be coalesced explicitly.

{\bf Map does not require FK enforcement for \tve.}  This is because
only \tav and \tae are affected, while the contents of \tv and \te
remain as in the input.


\section{Additional results.}
\label{sec:app2}

Plots and discussion in this section complement experimental results
presented in Section~\ref{sec:exp}.

