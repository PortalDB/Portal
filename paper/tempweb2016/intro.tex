\section{Introduction}
\label{sec:intro}

The development of SQL, a declarative query language for relational
data analysis, had a tremendous impact on the usability of database
technology, leading to its wide-spread adoption.  At the same time,
the separation between the logical and the physical representations
paved the way for powerful performance optimizations.  The motivation
for our research is to provide a similar tool for analyzing {\em
  evolving graphs}, an area of interest in many research communities,
including sociology, epidemiology, networking, etc.

Arguably the most interesting and important questions one can ask
about networks have to do with their evolution, rather than with their
static state.  Analysis of {\em evolving graphs} has been receiving
increasing attention, with most progress taking place in the last
decade~\cite{DBLP:journals/csur/AggarwalS14,Chan2008,Kan2009,DBLP:journals/tos/MiaoHLWYZPCC15,Ren2011,Semertzidis2015}.
Some areas where evolving graphs are being studied are social network
analysis~\cite{DBLP:conf/icwsm/GoetzLMF09,DBLP:journals/tweb/LeskovecAH07,DBLP:conf/kdd/LeskovecBKT08,DBLP:conf/icml/SarkarCJ12},
biological networks~\cite{DBLP:journals/tkdd/AsurPU09,DBLP:journals/tcsb/BeyerTLSF10,Stuart2003} and the Web~\cite{DBLP:journals/kais/ChanBL08,DBLP:journals/jisa/PapadimitriouDG10}.

Despite much recent interest and activity on the topic, and despite
increased variety and availability of evolving graph data, systematic
support for scalable querying and analytics over evolving graphs still
lacks. This support is urgently needed, due first and foremost to the
scalability and efficiency challenges inherent in evolving graph
analysis, but also to considerations of usability. 

 {\em In this paper we present our ongoing work on the \ql system, an
   open-source distributed framework that fills this gap.} \ql
 streamlines exploratory analysis of evolving graphs, making it
 efficient and usable, and providing critical tools to computational
 and data scientists.  Importantly, \ql implements a declarative query
 language for evolving graphs.  In what follows, we briefly describe
 the query language (Section~\ref{sec:language}) and the \ql system
 (Section~\ref{sec:sys}) that implements it.  We then describe several
 physical representations of evolving graphs that we developed
 (Section~\ref{sec:physical}), and show results of a preliminary
 experimental evaluation (Section~\ref{sec:exp}), which
 illustrates interesting performance trade-offs when different
 physical representations are used for different analytics.


\eat{\subsection{Introducing Portal}

Let us consider some categories of questions one may ask about
evolving networks in our system.

{\em Can information about graph evolution be used to make graph
  analytics more stable, or representative?}  Algorithms that compute
website popularly can be vulnerable to link spam, which is a
persistent phenomenon on the Web, but the identity of spammers is
transient~\cite{DBLP:conf/cikm/YangQZGL07}.  This suggests that
persistence vs. transience of a node, edge, or, more generally, of a
subgraph, is a meaningful aspect of quality.  Stable or representative
subgraphs have also been used to improve performance of iterative
computations in evolving graphs, e.g., for computing shortest
paths~\cite{Ren2011}.  The {\em temporal aggregation} operation in \ql
can be used to find a representative subgraph of an evolving graph.

{\em How can multiple data sources be used jointly, to complement or
  corroborate information about graph evolution?}  It may be the case
that multiple datasets are available, each describing a series of
crawls of different, but possibly overlapping, portions of the Web
graph.  Further, network states may be recorded at different, possibly
overlapping, time periods, or even at different temporal scales.  Can
these datasets be unified, in a principled way, to support analysis or
meta-analysis of network evolution trends?  \ql supports several
variants of the {\em temporal join operator} to enable this kind of
analysis.

{\em Have any changes in network connectivity been observed, either
  suddenly or gradually over time?}  For networks describing
insulin-based metabolism pathways, gradual pathway disruption can be
used to determine the onset of type-2
diabetes~\cite{DBLP:journals/tcsb/BeyerTLSF10}.  For a website
accessibility network, sudden loss of connectivity can signal that
censorship is taking place, e.g., in response to a recent election or
another exogenous event.  In a co-authorship network, increasing
connectivity among topical communities indicates stronger
collaboration across domains.  Here, again, connectivity can be
quantified as, e.g., pair-wise distance, length of shortest path
between communities, or graph density.  \ql supports this kind of
analysis with {\em snapshot and trend analytics}.

In the remainder of this demonstration proposal, we describe \ql
system design.  We also describe in detail how we plan to demonstrate
\ql at SIGMOD.}
