\section{Experimental Evaluation}
\label{sec:exp}

{\bf Experimental environment.} All experiments in this section were
conducted on an 8-slave in-house Open Stack cloud, using Linux Ubuntu
14.04 and Spark v1.4.  Each node has 4 cores and 16 GB of RAM.  Spark
Standalone cluster manager and Hadoop 2.6 were used.

Because Spark is a lazy evaluation system, a \insql{materialize}
operation was appended to the end of each query, which consisted of
the count of nodes and edges.  In cases where the goal was to evaluate
a specific operation in isolation, we used warm start, which consisted
of materializing the graph upon load.  Each experiment was conducted 3
times, we report the average running time, which is representative
because we took great care to control variability.  Standard deviation
for each measure is at or below 5\% of the mean except in cases of
very small running times.

{\bf Data.}  We evaluate performance of our framework on two real
open-source datasets.
%\begin{enumerate}%[leftmargin=*]
%\item 
DBLP~\cite{dblp} contains co-authorship information from 1936 through
2015, with over 1.5 million author nodes and over 6 million undirected
co-authorship edges.  Total data size: 250 MB.
%
nGrams~\cite{nGrams} contains word co-occurrence information from 1520
through 2008, with over 1.5 million word nodes and over 65 million
undirected co-occurrence edges.  Total data size: 40 GB.  

The nGrams dataset is of comparable size to the LiveJournal dataset
in~\cite{Xin2013} and is commensurate with our cluster size.  DBLP and
nGrams differ not only in size, but also in the evolutionary
properties: co-authorship network nodes and edges have limited
lifespan, while the nGrams network grows over time, with nodes and
edges persisting for long duration.  All figures in the body of this
section are on the larger nGrams dataset.  Refer to the Appendix for
the DBLP figures, which show similar trends as nGrams.  We plan to
carry out further experiments with a larger
DELIS\footnote{\url{law.di.unimi.it/webdata/uk-union-2006-06-2007-05}}
dataset as we grow the cluster in the near future.
