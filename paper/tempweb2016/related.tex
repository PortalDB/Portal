\section{Related Work}
\label{sec:related}

\eat{The research related to our \ql system is in the areas of queries for
evolving graphs, efficient evolving graph representation, on disk or
in memory, and graph partitioning.}

{\bf Querying and analytics.}  There has been much recent activity in
developing analytics for evolving graphs, see~\cite{} for a recent
survey.  This work is synergistic with ours, in that we build a
platform that supports efficient execution of a variety of existing
analytics.

\ql is the first proposed declarative query language for evolving
graphs.  Several researchers have proposed individual queries, without
the unifying syntax or generality.  For example, Kan, et
al.~\cite{Kan2009} propose a query model for evolving graphs, where
queries are for discovering matches for a specific spatio-temporal
pattern.  The only operator in this model is selection, with two
inputs: a temporal pattern and an arbitrary predicate.  \eat{The
  patterns are represented as waveforms (based on work by Chan et
  al.~\cite{Chan2008}) and the result of a query is a set of subgraphs
  in which all edges approximately follow the waveform provided.
  Here, similar to other mining works, the focus is topographical and
  the set of vertices remains unchanged through the evolution. Only
  changes are edge additions and deletions.  Portal's selection query
  is temporal-only and returns a single evolving graph, but the
  underlying API also allows structural selections with vertex and
  edge predicates.  However, the idea of using waveforms to represent
  structurally-temporal patterns is a reasonable future extension of
  our work.  The model for the evolving graph here is similar to our
  OneGraph, with a difference that we store absence or presence of not
  only edges but also vertices.  We also represent vertex and edge
  attributes, which are not modeled here.} Semertzidis, et
al. ~\cite{Semertzidis2015} also provide querying of evolving graphs,
with the focus on building indexes to support historical reachability
queries.  The most extensive system to date for querying of evolving
graphs is ImmortalGraph~\cite{DBLP:journals/tos/MiaoHLWYZPCC15}, an
in-memory execution engine for temporal graph analytics.
ImmortalGraph distinguishes between queries that do one-time
random-storage IO (query vertex/edge and query vertex/edge changes)
and repeated graph traversals in memory such as analytics.  \eat{For
  random-storage IO retrievals, physical data layout is important.
  The snapshot group method used by the authors is very similar to
  what Khurana~\cite{Khurana2013} did for temporal graph storage.
  Graph partitioning is also explored in this work, mainly exploring
  the tradeoff of temporal and spatial co-location, albeit on disk.
  It is shown that some queries benefit from temporal partitioning,
  while others from spatial one.  As a result, the Immortal graph
  keeps different type replicas on disk and picks the better one for
  the type of query requested.  Since our focus in this work is on
  in-memory query evaluation and optimization, the physical data
  layout is of less importance, but we also explore the tradeoffs of
  spatial and temporal partitioning, as well as hybrid strategies.
  Observe that since different data structures and partition
  strategies provide the best performance depending on the operation,
  the on-disk data itself representation is less critical.  The
  batched approach to analytics in this paper relies on the
  observation that most realworld graphs have a high unchanged ratio -
  the ratio of edges that each snapshot inherits from its immediate
  predecessor.  This observation may not hold at different levels of
  aggregation, especially with universal semantics, as we showed
  in~\ref{sec:experiments}.}  Our contribution is an
implementation-independent declarative language to support both
iterative analytics and building-block operations such as joins and
aggregations.

There are other areas that are related to our work but not closely.
For example, this work is different than work on query processing of
dynamic graphs (e.g., ~\cite{Mondal2012}), where the history of
changes is not important, and the focus is on updating the results as
the graph undergoes changes.  Similarly, work on querying evolving
graphs is different than the problem of mining evolving graph streams
(e.g., ~\cite{Liu2010}), where the focus is on discovery of
significant changes over a small window of consecutive graphs.

{\bf Data representation.}  Another important area of research is
efficient storage, retrieval, and in-memory representation of evolving
graphs.  Various variants of delta-based storage have been
investigated in the literature~\cite{Khurana2003,
  DBLP:journals/tos/MiaoHLWYZPCC15, Koloniari2012} and a combination
of deltas with selected materialized snapshots appears to provide the
best balance of performance and compactness.  Most notably, Khurana
and Deshpande~\cite{Khurana2003} investigate efficient physical
representations using deltas for a variety of datasets to support
snapshot retrieval, including different cases of skew. \eat{Another
  aspect relevant to our work is that of data skew and differential
  functions.  In DeltaGraph, a differential function specifies how the
  desired snapshot should be constructed from its children.  Depending
  on the nature of the graph evolution over time, different
  differential functions provide the best performance.  For example,
  for a growing-only graph, a mixed function provides better
  performance than Intersection.  This insight is useful for temporal
  graph partitioning, as allocating equal-sized partitions to each
  temporal interval could similarly lead to subpar performance.  In
  order to optimize performance, we need to precompute graph evolution
  statistics such as rate of change.  We leave this for future work.}
Miao, et al.~\cite{DBLP:journals/tos/MiaoHLWYZPCC15} use a similar
approach, but demonstrate that there is a tradeoff between temporal
and spatial co-location format based on the type of query.  For
example, large temporal range queries benefit from the time-locality
layout in their experiments.

Hybrid delta-snapshot approach provides an ability to support an
arbitrary evolving graph resolution.  \ql system basic building block
is a snapshot with the limitation that only resolutions no smaller
than the ones at which snapshots are taken can be computed.  It would
be fairly easy, however, to modify our system to store data as deltas
rather than snapshots, using Khurana's approach.  \eat{There are two
  other areas of his work that are relevant to ours.}  Boldi et
al.~\cite{Boldi2008} present a space-efficient non-delta approach for
storing a large evolving web graph that they harvested.  Their
approach for encoding the presence or absence of nodes/edges at each
time interval using bits is similar to our MGC data structure and use
of BitSets.  The primary difference is that their work represents
purely topological information and does not address vertex and edge
attributes, which still need to be represented and accessed by some
queries.  \eat{It is also notable that no single data structure is
  most optimal for supporting all the operations of the Portal
  language.}

In-memory evolving graph representation work in the literature aligns
with the data structures we described in
Section~\ref{sec:sys:datastructs}.  For example, Ren, et
al.~\cite{Ren2011} compute and store representative graphs, whch are
equal to our All and Any aggregations over snapshot clusters.  The
main approach is to store representative graphs in memory for each
cluster rather than all the snapshots, and to store deltas from
representative graphs to each snapshot if access to an individual
snapshot is required.  A further optimization is to store deltas to
only the first snapshot in the cluster, and then changes to successive
snapshots from the first instead of deltas.  This is similar to the
change tree approach in~\cite{Khurana2013}, although for memory-based
representation.  With a further optimization of exploring
inter-cluster similarity, the total storage savings are large
(reported only 0.86\% of a raw dataset needed in one example).  The
OneGraph data structure we employ can be thought of as a
representative graph for the whole selected time period.  The
GraphPool in-memory graph storage~\cite{Khurana2013} maintains a
single graph representing all retrieved snapshots, and is thus similar
to our MultiGraph and OneGraph data structures.  The GraphPool goes
further and stores only dependencies from a materialized snapshot in
those cases where the deltas between two snapshots are small.  Our
implementation does not take that step because evaluation of queries
involving multiple snapshots, such as TGroup, requires fully
materialized views in memory.  TimeReach~\cite{Semertzidis2015} uses a
representation called a version graph, where each node and edge is
annotated with the set of time intervals which which it existed.  This
representation is similar to our OneGraph data structure, with the
difference that we also store non-topological attribute information
and OneGraph does not compress consecutive existence intervals with
the same value into one.  While this optimization is compact, it makes
some queries, such as TGroup and snapshot analytics, highly complex.

\eat{Ren, et al.~\cite{Ren2011} propose an evolving graph query processing
framework based on the Find-Verify-and-Fix approach similar to
iterative methods.  Graph snapshots are clustered and representative
graphs computed for each cluster.  These representative graphs are
equal to our All and Any TGroup query results, which can then be used
to more efficiently compute shortest path queries.  This serves as
further validation of the necessity of the TGroup operation in Portal.
This paper also considers different storage models.  In the
future, we will expore whether a hybrid MG-SGP approach could provide
some benefits since it naturally supports many of the Portal
operations, topographically speaking.}

{\bf Distributed frameworks, partitioning.}  In this paper we build
upon, and non-trivially extend, the graph processing abstractions of
Apache Spark, a popular open-source distributed data processing
engine, and specifically of
GraphX~\cite{DBLP:conf/osdi/GonzalezXDCFS14}.  GraphX provides an API
for working with regular graphs (snapshots), without the time
dimension.  As described in~\ref{sec:system}, we build our TGraph data
structures on top of GraphX Graph, and provide a Portal language
interpreter in addition to scala API.  In order to provide better
performance, we modified GraphX graph loading code to enable
concurrent distributed multi-file loading with tuned number of
partitions.  Default number of partitions used to read in graphs, in
contrast, is slow and, for large graphs, leads to OOM errors.

Distributed graph computation requires balanced graph partitioning
dependent on the nature of the operation being applied to the graph.
There are two basic types of graph partitions: edge-cut and
vertex-cut.  An edge-cut approach distributes vertices across the
available machines and replicates the edges as necessary.  A
vertex-cut approach does the opposite.  Based on extensive theoretical
and experimental~\cite{Gonzalez2012} studies, GraphX chose the
vertex-cut approach, which minimizes the communication overhead.  The
first GraphX paper~\cite{Xin2013} provides an in-depth explanation of
how this is supported.  One of the built-in partition strategies, Edge
2D, provides a performance guarantee as a maximum number of
replications of a vertex.  We developed additional partition
strategies (all vertex-cut) based on the temporal dimension as well as
hybrids of structural and temporal.

Another framework build on top of Spark is Shark (now
SparkSQL)~\cite{Xin2013Shark}.  Shark provides mid-query optimization
of the distributed execution plan in addition to other optimizations
over spark such as better data representation.  Query optimization is
one of our goals for the Portal system, starting with dynamic
adjustment of the number of partitions and the partition strategy, but
that work is still in its infancy and we do not report on it here.
One immediately relevant aspect of the Shark work is in partitioning.
Namely, in Shark tables can be copartitioned on a particular key at
creation time which facilitates much more efficient joins.  This is
similar to what some of our partition strategies do for tgroup and
temporal joins to reduce communication overhead.

