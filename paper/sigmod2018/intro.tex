\section{Introduction}
\label{sec:intro}

Many social structures and systems can be represented as networks or
graphs.  The phenomena that are represented by these graphs can change
over time, some continuously and others sporadically.  Many
interesting questions about these networks are related to their
evolution rather than their static state.  Researchers study graph
evolution rate and mechanisms,
e.g.,~\cite{DBLP:journals/csur/AggarwalS14,Cho2000}), impact of
specific events on further evolution, e.g.,~\cite{Chan2008}), spatial
and spatio-temporal patterns, e.g.,~\cite{Lahiri2008}), with most
progress taking place in the last
decade~\cite{Kan2009,Miao2015,Ren2011,Semertzidis2015}.

Evolving graphs have different levels of granularity typical for
static graphs -- we may be interested in properties and behavior of
individual nodes, communities, and whole graphs.  At the same time,
the evolving graphs have an additional dimension that can have varying
granularity -- time.  It may be useful to zoom out in time by changing
the time granularity of the data such that all durations are, for
example, in months or years.  This way more items are concurrent and
the overall network is more connected.  We call this kind of zooming
out a change in {\bf temporal resolution} of the graph and we can use
it as part of exploratory analysis.

It is useful to zoom out in time and in structure in exploratory
analysis to see patterns at different levels that may not be visible
otherwise.  For example, an interaction network is one typical kind of
an evolving graph.  It represents people as nodes, and interactions
between them such as messages, conversations, and endorsements, as
edges.  Information describing people and their interactions is
represented by node and edge attributes.  Interaction networks are
sparse because edges are so short-lived.  In sparse interaction graphs
there is the question of temporal resolution to consider: if two
people communicated on, say, May 16, 2010, how long do we consider
them to be connected?  To see whether communities form and at what
time scale, we may vary the time scale and compute communities, e.g.,
through connected components detection, group the nodes by the
community they form.  We may also create a single node to represent
the whole graph to support computation of some whole-graph measures
such as graph centrality.

In this paper we propose two temporal graph algebra operations that
support analysis of evolving graphs at different temporal and
structural resolution: attribute-based node creation and window-based
node creation.  We discuss the semantics of both operators and their
implementation in a distributed environment.  We also provide an
evaluation of the operators on real datasets and compare their
performance to a baseline, where possible.




