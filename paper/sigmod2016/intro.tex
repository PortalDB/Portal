\section{Introduction}
\label{sec:intro}

The importance of networks cannot be overstated. Networks arise in a
plethora of domains, and are central to many applications, both
commercial and scientific.  Networks are represented by graphs, and
considerable research and engineering effort is being devoted to
developing effective and efficient graph representations and
analytics. (We will use the terms ``network'' and ``graph''
interchangeably.) As of late, efficient graph abstractions and
analytics for {\em static graphs} have become available in scope of
open source platforms such as Apache Spark (through the GraphX API)
and GraphLab (through the PowerGraph library).  This in turn makes
sophisticated graph analysis methods available and accessible to
researchers and practitioners, facilitating their widespread adoption.
Further, because these systems are open source, this encourages
development and dissemination of new graph analysis methods, and of
more efficient implementations of existing methods.

Arguably the most interesting and important questions one can ask
about networks have to do with their evolution, rather than with their
static state.  Analysis of {\em evolving graphs} has been receiving
increasing attention, with most progress taking place in the last
decade, see~\cite{DBLP:journals/csur/AggarwalS14} for a recent survey.
{\em Yet, systematic support for scalable querying and analytics over
  evolving graphs still lacks}.  This support is urgently needed, due
first and foremost to the scalability and efficiency challenges
inherent in evolving graph analysis, but also to considerations of
usability and ease of dissemination. {\em In this paper we present
  \ql, a declarative query language, and its implementation in an
  open-source distributed framework, that fills this gap.}

To further motivate our work, let us consider several categories of
questions one may ask about evolving networks, as part of exploratory
analysis.

{\em Which network nodes are showing an increasing popularity trend,
  or have increasing influence, and which are on a downward spiral?}
On the Web graph this information can help prioritize crawling.  In
social networks, it can be used for content recommendation and
advertisement targeting.  In semantic knowledge bases this information
can be used to capture the dynamics of zeitgeist.  Here, node
popularity can be quantified in a number of ways including, e.g., node
degree, centrality or PageRank score.

{\em Have any changes in network connectivity been observed, either
  suddenly or gradually over time?}  For networks describing
insulin-based metabolism pathways, gradual pathway disruption can be
used to determine the onset of type-2 diabetes.  For a website
accessibility network, sudden loss of connectivity can signal that
censorship is taking place, e.g., in response to a recent election or
another exogenous event.  In a co-authorship network, increasing
connectivity among topical communities indicates stronger
collaboration across domains.  Here, again, connectivity can be
quantified as, e.g., pair-wise distance, length of shortest path
between communities or graph density.

{\em At what time scale can interesting trends be observed?} The
answer to this question may not be known apriori, at the time when
graph evolution data is being recorded.  For example, changes in node
centrality in a social network may be observable on the scale of
weeks, but not months (coarser).  On the Web, periodic events may
change popularity of websites, with observable trends on the scale of
days, but not hours (finer) or months (coarser).  Furthermore, the
same network may show different kinds of trends at different time
scales, e.g., node popularity may be changing at a different rate, and
thus be observable on a different scale, than over-all network
density.  Understanding at what temporal scale to consider network
evolution is an integral part of exploratory analysis.

{\em How can multiple data sources be used jointly, to complement or
  corroborate information about graph evolution?}  It may be the case
that multiple datasets are available, each describing a series of
crawls of different, but possibly overlapping, portions of the Web
graph.  Further, network states may be recorded at different, possibly
overlapping, time periods, or even at different temporal scales.  Can
these datasets be unified, in a principled way, to support analysis or
meta-analysis of network evolution trends? 

As a preview of the development of this paper, consider the following
query.

\begin{verbatim}
TSelect   Any V [vid, any(name), pagerank() as pr]; 
          Any E [id1, id2, sum(cnt)]
From      T1 TAnd T2
Start     2010 End 2012
TGroup    by 2 years
\end{verbatim}

In the remainder of this paper we will explain precisely what this
query computes, and how it computes its result.

{\bf Contributions.} In this paper we develop \ql, a declarative query
language for efficient querying and exploratory analysis of evolving
graphs. Our language supports novel operations including temporal
aggregation, union and intersection, and a rich class of analytics.
Further, we provide a scalable and extensible open-source
implementation of \ql in scope of Apache Spark, an open-source
distributed data processing framework.  We develop several novel
physical representations of evolving graphs, and novel partitioning
strategies that explore the trade-off between structural and temporal
locality.  We experimentally demonstrate that good performance can be
achieved with careful engineering.

{\bf Roadmap.}  We present our model in Section~\ref{sec:model}.  We
illustrate the salient features of the \ql query language by example
in Section~\ref{sec:example}, and describe the algebraic operators of
in Section~\ref{sec:lang}.  \ql is fully implemented in an open-source
prototype, which is presented in Section~\ref{sec:sys}.
Section~\ref{sec:exp} describes our extensive experimental evaluation
on a variety of real datasets.  We discuss related work in
Section~\ref{sec:related}.  Future work and conclusions are given in
Section~\ref{sec:conc}.
