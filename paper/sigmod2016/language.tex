\section{TGraph Algebra}
\label{sec:lang}

The \ql query language is declarative.  We illustrated the syntax of
\ql in Section~\ref{sec:example}.  In this section we present
algebraic operators that implement \ql.  We support unary and binary
operators that take \tgs as input and produce a \tg as output.  Thus,
the language is fully compositional.  We will use the graph in
Figure~\ref{fig:sg} to illustrate query language operators.

\paragraph*{T-Select} The unary operation \tsel{T}{c}, where $T$ is
 a \tg and $c$ is a {\em temporal predicate}, returns a \tg $T'$
 containing a subset of the snapshots of $T$.  We adopt SQL 2011
 temporal predicate definitions for {\em contains}, {\em overlaps},
 {\em equals}, {\em preceeds}, and {\em succeeds} for time periods.
 Our use of these predicates will be illustrated as appropriate.  Note
 that if $c$ matches non-consecutive snapshots in $T$, $T'$ will still
 cover a consecutive time period, but some of the snapshots in $T'$
 will be empty.  For example, if $c$ matches months with names
 containing the letter 'r', then $T'$ will contain the snapshots for
 $T$ for January through April and September through December, and
 empty snapshots for May through August.

\paragraph*{S-Select} The unary operation \ssel{T}{c}, where $T$ is a \tg 
and $c$ is a {\em structural predicate}, returns a \tg $T'$ that has
the same temporal schema as $T$, and $\forall i \leq T.size, T'.G_i.V
\subseteq T.G_i.V \wedge T'.G_i.E \subseteq T.G_i.E$.  \julia{What is
  the right terminology for what I'm calling ``conjunctive structural
  predicate''?  Is there a standard way to state the obvious condition
  that vertices are checed first, and if a vertex disappears, then so
  do its edges?}

\paragraph*{U-Group} The unary operation \andgrp{T}{c}{\gamma_v}{\gamma_e}, 
where $T$ is a \tg, $c$ is an interval expression, $\gamma_v$ is a
vertex aggregation function, and $\gamma_e$ is an edge aggregation
function, groups snapshots of $T$ based on $c$, and then aggregates
the vertices and edges of snapshots in the same group by performing
{\em structural union} of the snapshots.  By structural union we mean
that a vertext will be present in the snapshot for group $g$ if it was
present in at least one snapshot that maps to $g$.  And edge will be
present in the aggregate if both vertices of that edge are present,
and if the edge was present in at least one snapshot that mapts to
$g$.

Functions $\gamma_v$ and $\gamma_e$ specify how attributes of a
vertext (resp. edge) that occurs in multiple snapshots in a given
group are handled.  For example ...

\paragraph*{E-Group}  The unary operation \orgrp{T}{c}{\gamma_v}{\gamma_e}, 
with the same arguments as {\bf U-Group}, groups snapshots of $T$
based on $c$, and then aggregates the vertices and edges of snapshots
in the same group by performing {\em structural intersection} of the
snapshots.  By structural intersection we mean that a vertext
(resp. edge) will be present in the snapshot for group $g$ if it is
present in all snapshots that map to $g$.  If any of the snapshots
that map to $g$ are empty, then so is the snapshot for $g$.  Functions
$\gamma_v$ and $\gamma_e$ have the same meaning for {\bf U-Group};
they specify how attributes of vertices and edges are aggregated.  For
example...

\paragraph*{Transform} The unary operation \xfrm{T}{map_v}{map_e}, where $T$ 
is a \tg, $map_v$ is a vertext map function and $map_e$ is an edge map
function, transforms the structural schema of a \tg, while keeping its
temporal schema intact.  For example, ...

\paragraph*{Analytic} The unary operation \an{T}{f_v}, where $T$ is a \tg and $f$ 
is an analytic function, computes the value of an analytic function
for each vertex of $T$.  \julia{Do we want to say that this is
  restricted to the Pregel API?  Do we need an edge function as well,
  or just a vertext function?}  For example, ...

\paragraph*{T-Union} The binary operation \tor{T_1}{T_2}{\gamma_v}{\gamma_e} 
computes the union of union-compatible \tgs $T_1$ and $T_2$.  The
result $T$ has temporal schema $T_1.P \cup T_2.P$.  If $T_1.P \cap
T_2.P$ is non-empty, structural union of the corresponding snapshots
is taken, and $\gamma_v$ and $\gamma_e$ are applied to the to the
attributes of the vertices (resp. edges) that are in common.  \julia{I
  don't like that this is restricted to structural union.  Is there a
  way to argue for this, or do we want to also support structural
  intersection as part of T-Union?  The same question for
  T-Intersect.}  For example...

\paragraph*{T-Intersect} The binary operation \tand{T_1}{T_2}{\gamma_v}{\gamma_e} 
computes the intersection of union-compatible \tgs $T_1$ and $T_2$.
The resulting \tg has temporal schema $T_1.P \cap T_2.P$, and each
snapshot is the result of structural intersection of the corresponding
snapshots, with $\gamma_v$ and $\gamma_e$ applied to the attributes of
the vertices (resp. edges).

Our language supports both unary and binary operators. They are:
\begin{itemize}
\item T-Select -- select a temporal subset based on a temporal expression.
\item S-Select -- select a subgraph for every snapshot (selection over V and E).
\item Aggregation -- change the granularity of the graph based on predefined semantics.
\item Transform -- modify the structural schema of the graph (similar to a map operation in map-reduce).
\item Graph analytics -- compute an analytic over all snapshots of the graph.
\item Union -- compute union of two temporal graphs.
\item Intersection -- compute intersection of two temporal graphs.
\end{itemize}

\begin{definition} [T-Select]
\label{def:tsel}
A time selection operator $T-\sigma_\phi(T) = (G_x, \ldots, G_y; P')$ is
a temporal subset operator on $T$ with propositional formula $\phi$
over $P$, where $1 \leq x,y \leq n$, $P' \subset P$.  The
propositional formula is a time expression over $P$.
\end{definition}

Time select operator changes the temporal schema of graph $T$ while
keeping the structural schema intact.

Similar to ~\cite{Khurana2013}, we support the following types of time expressions:

\begin{itemize}

\item Time iterval.  Any valid time interval $[p.start,p.end)$ can be
  used for T-Select.  The selected temporal sequence is the interval
  intersection \ref{def:tseqii} of the input time interval and T's
  temporal sequence P.  I.e., only the overlapping portion of T's
  temporal sequence and input period p will be selected.

\item Interval predicate.  A predicate that can be applied to each
  time interval $p_i \in P$.  Because a temporal sequence does not
  allow for any gaps, empty snapshots are inserted in each time
  interval that does not meet the selection criteria.  \vera{I am not
    sure if I like this...}

\end{itemize}

\vera{Khurana also supports single time point selection, but I don't
  think this goes well with our model.}

For example, T-Select over graph T with temporal sequence
$[1,3),\ldots,[9,11)$ with argument [3,12) results in a $T'$ with $P'
      = [3,11)$.  In contrast, for the same graph T with argument
        [4,6) results in a null sequence $\emptyset$ because there is
          no valid overlap.

\begin{definition}
\label{def:ssel}
A structure select operator $S-\sigma_\phi(T)$ returns a structural
subgraph $T'= (G_1', \ldots, G_n'; P)$ on $T$, where $\forall i \in 1
\ldots n, V'_i \subseteq V_i$, $E'_i \subseteq E_i$. The temporal
schema of the temporal graph is not affected, i.e. $T.P = T'.P$.

\vera{But what happens if the subgraph operation on the last (or
  first) snapshot results in an empty snapshot? Should we tighten the
  interval so that P' is subset of P or keep P as is and have empty
  snapshots? What we have above is option b which I think is more
  consistent.}
\end{definition}

\begin{definition}
\label{def:agg}
An aggregation operator $G f(V_i) g(E_i) (T)$ returns a subgraph 

$T' = (G_1', \ldots, G_m', P')$, where $\forall G_i' = \cup(G_{i},
\cdots, G_{i+r'-1})$ for existential semantics and $\cap(G_{i},
\cdots, G_{i+r'-1})$ for universal semantics.  $f(V_i)$ is a function
applied to node attribute values of nodes with the same identity.
Similarly, $g(E_i)$ is a function applied to edge attribute values
with the same identity. \vera{We need to express somehow that if |GS|
  is not evenly divisible by r', we reduce or increase as needed to
  hold the size/resolution constraint.}
\end{definition}

Aggregation changes temporal schema of the graph, while keeping the
snapshot schema.

For example, consider the temporal graph in Figure~\ref{fig:agg}.

\begin{figure}
\label{fig:agg}
\caption{...}
\end{figure}

\begin{definition}
\label{def:transform}
A transform operator $map f(V_i) g(E_i) (T)$ is a modification
function on T which returns $T' = (G_1', \cdots, G_n', P)$ where
$\forall V_i' = f(V_i), E_i' = g(E_i)$.
\end{definition}

The transform operator keeps the graph temporal schema, as well as
snapshot structure, while modifying snapshot schema and/or values,
computed 1 node at a time.  This is similar to the map operation in
MapReduce.

