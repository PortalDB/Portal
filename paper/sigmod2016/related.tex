\section{Related Work}
\label{sec:related}

In this paper We will build upon, and non-trivially extend, the graph
processing abstractions of Apache Spark, a popular open-source
distributed data processing engine, and specifically of
GraphX~\cite{DBLP:conf/osdi/GonzalezXDCFS14}.  

Our work shares motivation with recent work by Miao et
al.~\cite{DBLP:journals/tos/MiaoHLWYZPCC15}, who developed an
in-memory execution engine for temporal graph analytics.  Immortal
Graph distinguishes between queries that do one-time random-storage IO
and repeated graph traversals in memory such as analytics.  For
random-storage IO retrievals, physical data layout is important.  The
snapshot group method used by the authors is very similar to what
Khurana~\cite{} did for temporal graph storage.  We, in contrast, do
not worry about physical data layout on disk and just deal with
snapshots of saved resolution.  However, our file format favors SGP
while MG and OG have to do expensive group-by operations.  Immortal
graph instead just keeps different type replicas on disk and picks the
better one for the type of query requested.  Also, another benefit of
this approach to physical storage is that it can support an arbitrary
resolution, while we can only support resolutions no smaller than the
ones at which snapshots are taken.

The batched approach to analytics in this paper relies on the
observation that most realworld graphs have a high unchanged ratio -
the ratio of edges that each snapshot inherits from its immediate
predecessor.  This observation may not hold at different levels of
aggregation, especially with universal semantics.

Amol's work: snapshot retrieval and anything else that is relevant.

The only work on defining a query language for evolving graphs:~\cite{Kan2009}.

Pattern mining work by Borgward~\cite{Borgwardt2006}, Chan~\cite{Chan2008}.
