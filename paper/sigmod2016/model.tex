\section{Model}
\label{sec:model}

In this section we first formally define snapshot graphs, and then
show how graph evolution is represented by assigning temporal meaning
to sequences of snapshot graphs.

\begin{definition}[Snapshot graph]
\label{def:sg} 
A {\em snapshot graph} (or a {\em snapshot}) is a pair $G = (V,E)$,
where $V$ is a finite set of nodes with schema $(\underline{vid},
a_1, \ldots, a_n)$, and $E$ is a finite set of edges connecting
pairs of nodes from $V$, with schema $(\underline{vid_1},
\underline{vid_2}, a_1, \ldots, a_m)$.
\end{definition}

Attributes of vertices and of edges are not restricted to be of atomic
types, but may, e.g., be maps or tuples. However it is required that
all vertices (resp. edges) of $G$ have the same schema, i.e., $V$ and
$E$ are homogeneous sets.

\begin{definition} [Structural union-compatibility]
\label{def:scompat}
Snapshot graphs$G' = (V', E')$ and $G'' = (V'', E'')$ are
union-compatible if $V'$ and $V''$ are union-compatible, and $E'$ and
$E''$ are union-compatible.
\end{definition}

$G$ may represent a directed or an undirected graph.  For undirected
graphs we choose a canonical representation of an edge, with $vid_1
\leq vid_2$ (self-loops are allowed).

We next describe how time is represented in our model.  Following the
SQL:2011 standard~\cite{DBLP:journals/sigmod/KulkarniM12}, we adopt
the {\em closed-open} period model, i.e., a period represents all
times starting from and including the start time, continuing to but
excluding the end time.

\begin{definition}[Time period]
\label{def:period} 
A {\em time period} \\$p = [start, end)$ is an interval on the timeline,
  subject to the constraint $start < end$.  We refer to the length of
  time covered by $p$ as its {\em resolution}.
\end{definition}

We focus on {\em valid time}, represented by {\em application-time
  period} in SQL:2011 --- the time period during which data is
regarded as correctly reflecting reality.  This is in contrast to {\em
  transaction time} (or {\em system-time period}), which refers to the
time period during which a row is committed to the database.  Our goal
in this work is to support complex analytics over evolving graphs,
under the assumption that all historical data is available in the
database and is read-only.

We represent graph evolution by associating a sequence of snapshots,
which are not themselves time-aware, with a a sequence of time
periods.  This is stated formally next.

\begin{definition} [Temporal sequence]
\label{def:tseq} 
A {\em temporal sequence} $P = (p_1, \ldots, p_n)$ is a
sequence of consecutive non-overlapping time periods of the same
resolution, with no gaps.  That is,

\begin{enumerate}
\item $\forall i < n, p_i.end = p_{i+1}.start$, and 
\item $\forall i, j, p_i.end - p_i.start = p_j.end - p_j.start$.
\end{enumerate}
\end{definition}

$P$ may be equivalently described by 3 values: the start of the
earliest period $P.start = p_1.start$, the end of the latest period
$P.end = p_n.end$, and the resolution of any period $P.res = p_1.end -
p_1.start$. For convenience, we refer to the number of periods in the
sequence as $P.size$.  

\julia{Define the null sequence: what are the start / end /
  resolution, considering that $start < end$, not $start \leq end$.}

\begin{definition} [Temporal Union-Compatibility]
\label{def:tcompat} 
Temporal sequences $P'$ and $P''$ are union-compatible if they have
the same resolution, and if we can construct a valid temporal sequence
$P$ with $P.start = min(P'.start, P''.start)$, $P.end = max(P'.start,
P''.start)$, and with $P.res = P'.res$.
\end{definition}

Recall that a snapshot represents a single state of an evolving graph,
and is not itself time-aware.  Temporal evolution of a graph is
represented by a sequence of snapshots, called {\em temporal graphs}
in our formalism. 

\begin{definition} [Temporal Graph]
\label{def:tgraph} 
A {\em temporal graph} $T = (G_1, \ldots, G_n; P)$ associates a
sequence of $n$ structurally union-compatible snapshots with a
temporal sequence $P$, such that $P.size = n$.
\end{definition}

Snapshot graphs in the sequence define the {\em structural schema} of
$T$, while the {\em temporal schema} of $T$ is specified by $P$.
Identity of a vertex persists across snapshots in a temporal graph,
and across temporal graphs.

A  temporal graph of Definition~\ref{def:tgraph} is the basic
element in our model.  In what follows, we assume that a relation in
our database corresponds to a single temporal graph, not to a
collection of temporal graphs.  In the next section we will define a
temporal graph query language \ql, in which operators take as input a
single temporal graph or a pair of temporal graphs, and produce a
temporal graph as output.

To support binary operations, we now define union compatibility of
temporal graphs.

\begin{definition} [Graph Union-Compatibility]
\label{def:tuc} Temporal graphs $T'$ and $T''$ are union-compatible if they are both
structurally union-compatible (per Definition~\ref{def:scompat}) and
temporally union-compatible (per Definition~\ref{def:tcompat}).
\end{definition}




