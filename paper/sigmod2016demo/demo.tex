\section{Demonstration Details}
\label{sec:demo}

We will present an end-to-end implementation of \ql and will
demonstrate that our system is usable and efficient, by showing that
sophisticated kinds of exploratory analysis of large evolving graphs
can be expressed intuitively and concisely, and executed in
interactive time.

\subsection{Demo Setup}
\label{sec:setup}

Users will interact with the \ql system via an interactive shell and
via the \qlui, where they will type in or visually compose queries and
define \tg views, exploring language features.\eat{ and its
  expressiveness.}  Users will inspect optimized query execution plans
and execute queries.

We will have a local cluster with three evolving graph datasets that
allow interactive-speed data exploration. DBLP~\cite{dblp} contains
co-authorship information from 1936 through 2015, with over 1.5
million author nodes and over 6 million undirected co-authorship
edges.  arXiv~\cite{arxiv} contains co-authorship information from
1993 through 2016, filtered to Computer Science publications.
nGrams~\cite{nGrams} lists word co-occurrences from 1520 through
2008, with over 1.5 million word nodes and over 65 million undirected
co-occurrence edges.  

\subsection{Story Line}
\label{sec:story}

\eat{We will demonstrate the functionality of \ql by executing queries
provided by us and by members of the audience. Queries on very
  large evolving graphs will not result in interactive running times
  on the stand-alone cluster we plan to use for the demonstration.
  However, the DBLP dataset that we will primarily use allows for
  exploratory querying that takes on the order of seconds on a small
  cluster.  The total running time is dominated by the time it takes
  to load data from HDFS, except in cases of using snapshot analytics
  over long time periods. Some queries on the nGrams dataset also can
  be done in interactive time.} 

{\bf Use Case 1:} DBLP dataset can provide interesting insights into
Computer Science research community at large.  Which authors have the
most consistent publication record, i.e. publish every year over a
long time period? This question can be answered using temporal
aggregation.  How stable are the co-authorship relationships over
time? This question can also be answered using temporal aggregation.
Who are the most authoritative sources, as judged by the co-authorship
relationships, and does that coincide with our own judgments?  This
question can be answered using pagerank analytic over an aggregation
of various durations since the by-year data is very sparse.  Members
of the audience may come up with other questions.  We will record all
the questions to see what language features, if any, are currently
missing from \ql but desirable to the community.

An interesting complex query computes the answer the following
question: Who are the top-10 researchers whose publication careers
exhibit the highest rising popularity over the last 20 years, using
both DBLP and arXiv datasets?  We can use the following SQL-\ql
query:

\begin{small}
\begin{verbatim}
 Q4: Select vid, pr
     From (TSelect Any V[vid, trend(prank) as pr];
                   Any E
           From (TSelect All V[vid, pagerank() as prank]; 
                         All E
                 From dblp TAnd arxiv
                 TWhere Start >= 1995-01-01 And
                        End < 2015-01-01
                 TGroup by 5 years)
           TGroup by size).toVerticesFlat()
     Order by pr
     Limit 10
\end{verbatim}
\end{small}

Note the use of \insql{Any} and \insql{All} modifiers in \insql{Q4}.
The \insql{Any} modifier makes explicit the default behavior of
structural aggregation, where a union of the vertices and edge is
computed.  \insql{All} specifies that the vertex and edge relations
should be intersected in the corresponding snapshots.

With this query users will not only get to experiment with the
available features of \ql, but also observe query optimization in
action.  We will explain that several different orders of operators
produce equivalent results for this query, but lead to different
performance.

{\bf Use Case 2:} Word co-occurrences in the nGrams dataset provide
interesting history of the evolution of the English language over 400
years.  Do co-occurrence pairs persist over time?  By experimenting
with different aggregation window sizes, we can see how stable
two-word phrases are long-term.  What words are the most popular in
each century, as judged by the number of phrases in which they appear?

An interesting query on this data answers the question: Which words
are becoming obsolete most rapidly over the past 50 years?  We can use
the following SQL-\ql query:

\begin{small}
\begin{verbatim}
 Q5: Select name, pop
     From (TSelect Any V[vid, trend(degree) as pop];
                   Any E
           From (TSelect V[vid, degree()]; 
                         E[vid1, vid2]
                 TWhere Start >= 1965-01-01)
           TGroup by size).toVerticesFlat()
     Order by pop asc
     Limit 10
\end{verbatim}
\end{small}

Users will explore the dataset and come up with their own questions,
and with \ql queries to answer those questions.

