\section{Introduction}
\label{sec:intro}

The development of SQL, a declarative query language for relational
data analysis, had a tremendous impact on the usability of database
technology, leading to its wide-spread adoption.  At the same time,
the separation between the logical and the physical representations
paved the way for powerful performance optimizations.  The motivation
for our research is to provide a similar tool for analyzing {\em
  evolving graphs}, an area of interest in many research communities,
including sociology, epidemiology, networking, etc.

Many phenomena can be represented as networks or graphs (two terms
used here interchangeably).  The Internet is often thought of as a
graph where Web pages are the nodes and hyperlinks between them are
the edges.  A social network is another commonly used graph type, with
people or organizations as nodes and their activities or relationships
as edges.  Less frequently mentioned but still very useful for
research are graphs formed from sensor and road networks, animal herds
(for the purpose of studying epidemics), metabolism pathways, and many
others.

The phenomena represented by these graphs change over time, some
continuously and others sporadically.  Many interesting questions
about these networks are related to their evolution rather than to
their static state.  Researchers study graph evolution rate and
mechanisms (e.g.,~\cite{DBLP:journals/csur/AggarwalS14,Cho2000}),
impact of specific events on further evolution
(e.g.,~\cite{Chan2008}), spatial and spatio-temporal patterns
(e.g.,~\cite{Lahiri2008}).  Many of these researchers are not computer
scientists and need system support to perform their data analysis.

Despite much recent interest and activity on the topic, and despite
increased variety and availability of evolving graph data, systematic
support for scalable querying and analytics over evolving graphs still
lacks. This support is urgently needed, due first and foremost to the
scalability and efficiency challenges inherent in evolving graph
analysis, but also to considerations of usability.  {\em In this
  demonstration we present \ql, an open-source distributed framework
  that fills this gap.} \ql streamlines exploratory analysis of
evolving graphs, making it efficient and usable, and providing
critical tools to computational and data scientists.

In what follows, we briefly describe our declarative query language
for evolving graphs (Section~\ref{sec:language}) and the \ql system
(Section~\ref{sec:sys}) that implements it.  We then present several
interesting demonstration scenarios (Section~\ref{sec:demo}). We
finish with a brief discussion of related work
(Section~\ref{sec:related}) and with take-home messages
(Section~\ref{sec:conc}).

\eat{\subsection{Introducing Portal}

Let us consider some categories of questions one may ask about
evolving networks in our system.

{\em Can information about graph evolution be used to make graph
  analytics more stable, or representative?}  Algorithms that compute
website popularly can be vulnerable to link spam, which is a
persistent phenomenon on the Web, but the identity of spammers is
transient~\cite{DBLP:conf/cikm/YangQZGL07}.  This suggests that
persistence vs. transience of a node, edge, or, more generally, of a
subgraph, is a meaningful aspect of quality.  Stable or representative
subgraphs have also been used to improve performance of iterative
computations in evolving graphs, e.g., for computing shortest
paths~\cite{Ren2011}.  The {\em temporal aggregation} operation in \ql
can be used to find a representative subgraph of an evolving graph.

{\em How can multiple data sources be used jointly, to complement or
  corroborate information about graph evolution?}  It may be the case
that multiple datasets are available, each describing a series of
crawls of different, but possibly overlapping, portions of the Web
graph.  Further, network states may be recorded at different, possibly
overlapping, time periods, or even at different temporal scales.  Can
these datasets be unified, in a principled way, to support analysis or
meta-analysis of network evolution trends?  \ql supports several
variants of the {\em temporal join operator} to enable this kind of
analysis.

{\em Have any changes in network connectivity been observed, either
  suddenly or gradually over time?}  For networks describing
insulin-based metabolism pathways, gradual pathway disruption can be
used to determine the onset of type-2
diabetes~\cite{DBLP:journals/tcsb/BeyerTLSF10}.  For a website
accessibility network, sudden loss of connectivity can signal that
censorship is taking place, e.g., in response to a recent election or
another exogenous event.  In a co-authorship network, increasing
connectivity among topical communities indicates stronger
collaboration across domains.  Here, again, connectivity can be
quantified as, e.g., pair-wise distance, length of shortest path
between communities, or graph density.  \ql supports this kind of
analysis with {\em snapshot and trend analytics}.

In the remainder of this demonstration proposal, we describe \ql
system design.  We also describe in detail how we plan to demonstrate
\ql at SIGMOD.}
