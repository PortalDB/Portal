\section{Introduction}
\label{sec:intro}

The development of the Structured Query Language (SQL) for relational
data analysis had a tremendous impact on the database community and
database users.  It gave a reasonably user-friendly tool for
non-programmers to perform querying and exploratory data analysis.
At the same time, its declarative nature separated the implementation
from the language itself, paving way for alternative implementations
and query optimization.  The motivation for this research is to
provide a similar tool for analyzing evolving graphs, an area of
interest in many research communities, including sociology,
epidemiology, networking, etc.

Many phenomena can be represented as networks or graphs (two terms
used here interchangeably).  The Internet, or portion thereof, is
often thought of as a graph where web pages or web sites are the graph
nodes, and the hyperlinks between them are the graph edges.  A social
network is another commonly used graph type, with people or
organizations as nodes and their activities or relationships serving
as the edges.  Less frequently mentioned but still very useful for
research are graphs formed from sensor and road networks, animal herds
(for the purpose of studying epidemics), metabolism pathways, and many
others.

The phenomena that are represented by these graphs change over time,
some continuously and others sporadically.  Many interesting questions
about these networks are related to their evolution rather than their
frozen moment in time state.  Researchers study graph evolution rate
and mechanisms (e.g.,~\cite{Cho2000,DBLP:journals/csur/AggarwalS14}),
impact of specific events on further evolution
(e.g.,~\cite{Chan2008}), spatial and spatiotemporal patterns
(e.g.,~\cite{Lahiri2008}).  As with database users, many of these
researchers are not computer scientists and they need system support
to perform their data analysis.

\subsection{Introducing Portal}

In this paper we present \ql, a declarative query language and system
for efficient querying and exploratory analysis of evolving graphs.
Our language supports a variety of operations including temporal
selection, join, and aggregation, and a rich class of analytics.  We
provide a scalable and extensible open-source implementation of \ql in
scope of Apache Spark, an open-source distributed data processing
framework.  Let us consider some categories of questions one may ask
about evolving networks in our system.

{\em Can information about graph evolution be used to make graph
  analytics more stable, or representative?}  Algorithms that compute
website popularly can be vulnerable to link spam, which is a
persistent phenomenon on the Web, but the identity of spammers is
transient~\cite{DBLP:conf/cikm/YangQZGL07}.  This suggests that
persistence vs. transience of a node, edge, or, more generally, of a
subgraph, is a meaningful aspect of quality.  Stable or representative
subgraphs have also been used to improve performance of iterative
computations in evolving graphs, e.g., for computing shortest
paths~\cite{Ren2011}.  The {\em temporal aggregation} operation in \ql
can be used to find a representative subgraph of an evolving graph.

{\em How can multiple data sources be used jointly, to complement or
  corroborate information about graph evolution?}  It may be the case
that multiple datasets are available, each describing a series of
crawls of different, but possibly overlapping, portions of the Web
graph.  Further, network states may be recorded at different, possibly
overlapping, time periods, or even at different temporal scales.  Can
these datasets be unified, in a principled way, to support analysis or
meta-analysis of network evolution trends?  \ql supports several
variants of the {\em temporal join operator} to enable this kind of
analysis.

{\em Have any changes in network connectivity been observed, either
  suddenly or gradually over time?}  For networks describing
insulin-based metabolism pathways, gradual pathway disruption can be
used to determine the onset of type-2
diabetes~\cite{DBLP:journals/tcsb/BeyerTLSF10}.  For a website
accessibility network, sudden loss of connectivity can signal that
censorship is taking place, e.g., in response to a recent election or
another exogenous event.  In a co-authorship network, increasing
connectivity among topical communities indicates stronger
collaboration across domains.  Here, again, connectivity can be
quantified as, e.g., pair-wise distance, length of shortest path
between communities, or graph density.  \ql supports this kind of
analysis with {\em snapshot and trend analytics}.

In the remainder of this demonstration proposal, we describe \ql
system design.  We also describe in detail how we plan to demonstrate
\ql at SIGMOD.
