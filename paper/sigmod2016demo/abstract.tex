\begin{abstract}

Graphs are used to represent a plethora of phenomena, from the Web and
social networks, to biological pathways, to semantic knowledge
bases. Arguably the most interesting and important questions one can
ask about graphs have to do with their evolution. Which Web pages are
showing an increasing popularity trend? How does influence propagate
in social networks? How does knowledge evolve?  

In this demonstration we present \ql, a system for efficient querying
and exploratory analysis of evolving graphs, built on Apache Spark.
The \ql system implements a declarative query language by the same
name.  We provide both a text-based query shell and a visual query
composer for \ql.  We demonstrate that our system is usable and
efficient, by showing that sophisticated kinds of exploratory analysis
of large evolving graphs can be expressed intuitively and concisely,
and executed in interactive time.

\eat{\ql is a declarative query language for evolving graphs and a system
built on Apache Spark.  Our system supports a variety of operations
including temporal selection, join and aggregation, and a rich class
of analytics.  We provide both a text-based query shell and a
graphical user interface for query composition and evaluation
targeting exploratory data analysis.}

\end{abstract}
